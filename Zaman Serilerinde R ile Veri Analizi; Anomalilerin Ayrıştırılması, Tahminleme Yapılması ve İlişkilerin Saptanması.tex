
% Default to the notebook output style

    


% Inherit from the specified cell style.




    
\documentclass[11pt]{article}

    
    
    \usepackage[T1]{fontenc}
    % Nicer default font (+ math font) than Computer Modern for most use cases
    \usepackage{mathpazo}
    % Basic figure setup, for now with no caption control since it's done
    % automatically by Pandoc (which extracts ![](path) syntax from Markdown).
    \usepackage{graphicx}
    % We will generate all images so they have a width \maxwidth. This means
    % that they will get their normal width if they fit onto the page, but
    % are scaled down if they would overflow the margins.
    \makeatletter
    \def\maxwidth{\ifdim\Gin@nat@width>\linewidth\linewidth
    \else\Gin@nat@width\fi}
    \makeatother
    \let\Oldincludegraphics\includegraphics
    % Set max figure width to be 80% of text width, for now hardcoded.
    \renewcommand{\includegraphics}[1]{\Oldincludegraphics[width=.8\maxwidth]{#1}}
    % Ensure that by default, figures have no caption (until we provide a
    % proper Figure object with a Caption API and a way to capture that
    % in the conversion process - todo).
    \usepackage{caption}
    \DeclareCaptionLabelFormat{nolabel}{}
    \captionsetup{labelformat=nolabel}

    \usepackage{adjustbox} % Used to constrain images to a maximum size 
    \usepackage{xcolor} % Allow colors to be defined
    \usepackage{enumerate} % Needed for markdown enumerations to work
    \usepackage{geometry} % Used to adjust the document margins
    \usepackage{amsmath} % Equations
    \usepackage{amssymb} % Equations
    \usepackage{textcomp} % defines textquotesingle
    % Hack from http://tex.stackexchange.com/a/47451/13684:
    \AtBeginDocument{%
        \def\PYZsq{\textquotesingle}% Upright quotes in Pygmentized code
    }
    \usepackage{upquote} % Upright quotes for verbatim code
    \usepackage{eurosym} % defines \euro
    \usepackage[mathletters]{ucs} % Extended unicode (utf-8) support
    \usepackage[utf8x]{inputenc} % Allow utf-8 characters in the tex document
    \usepackage{fancyvrb} % verbatim replacement that allows latex
    \usepackage{grffile} % extends the file name processing of package graphics 
                         % to support a larger range 
    % The hyperref package gives us a pdf with properly built
    % internal navigation ('pdf bookmarks' for the table of contents,
    % internal cross-reference links, web links for URLs, etc.)
    \usepackage{hyperref}
    \usepackage{longtable} % longtable support required by pandoc >1.10
    \usepackage{booktabs}  % table support for pandoc > 1.12.2
    \usepackage[inline]{enumitem} % IRkernel/repr support (it uses the enumerate* environment)
    \usepackage[normalem]{ulem} % ulem is needed to support strikethroughs (\sout)
                                % normalem makes italics be italics, not underlines
    

    
    
    % Colors for the hyperref package
    \definecolor{urlcolor}{rgb}{0,.145,.698}
    \definecolor{linkcolor}{rgb}{.71,0.21,0.01}
    \definecolor{citecolor}{rgb}{.12,.54,.11}

    % ANSI colors
    \definecolor{ansi-black}{HTML}{3E424D}
    \definecolor{ansi-black-intense}{HTML}{282C36}
    \definecolor{ansi-red}{HTML}{E75C58}
    \definecolor{ansi-red-intense}{HTML}{B22B31}
    \definecolor{ansi-green}{HTML}{00A250}
    \definecolor{ansi-green-intense}{HTML}{007427}
    \definecolor{ansi-yellow}{HTML}{DDB62B}
    \definecolor{ansi-yellow-intense}{HTML}{B27D12}
    \definecolor{ansi-blue}{HTML}{208FFB}
    \definecolor{ansi-blue-intense}{HTML}{0065CA}
    \definecolor{ansi-magenta}{HTML}{D160C4}
    \definecolor{ansi-magenta-intense}{HTML}{A03196}
    \definecolor{ansi-cyan}{HTML}{60C6C8}
    \definecolor{ansi-cyan-intense}{HTML}{258F8F}
    \definecolor{ansi-white}{HTML}{C5C1B4}
    \definecolor{ansi-white-intense}{HTML}{A1A6B2}

    % commands and environments needed by pandoc snippets
    % extracted from the output of `pandoc -s`
    \providecommand{\tightlist}{%
      \setlength{\itemsep}{0pt}\setlength{\parskip}{0pt}}
    \DefineVerbatimEnvironment{Highlighting}{Verbatim}{commandchars=\\\{\}}
    % Add ',fontsize=\small' for more characters per line
    \newenvironment{Shaded}{}{}
    \newcommand{\KeywordTok}[1]{\textcolor[rgb]{0.00,0.44,0.13}{\textbf{{#1}}}}
    \newcommand{\DataTypeTok}[1]{\textcolor[rgb]{0.56,0.13,0.00}{{#1}}}
    \newcommand{\DecValTok}[1]{\textcolor[rgb]{0.25,0.63,0.44}{{#1}}}
    \newcommand{\BaseNTok}[1]{\textcolor[rgb]{0.25,0.63,0.44}{{#1}}}
    \newcommand{\FloatTok}[1]{\textcolor[rgb]{0.25,0.63,0.44}{{#1}}}
    \newcommand{\CharTok}[1]{\textcolor[rgb]{0.25,0.44,0.63}{{#1}}}
    \newcommand{\StringTok}[1]{\textcolor[rgb]{0.25,0.44,0.63}{{#1}}}
    \newcommand{\CommentTok}[1]{\textcolor[rgb]{0.38,0.63,0.69}{\textit{{#1}}}}
    \newcommand{\OtherTok}[1]{\textcolor[rgb]{0.00,0.44,0.13}{{#1}}}
    \newcommand{\AlertTok}[1]{\textcolor[rgb]{1.00,0.00,0.00}{\textbf{{#1}}}}
    \newcommand{\FunctionTok}[1]{\textcolor[rgb]{0.02,0.16,0.49}{{#1}}}
    \newcommand{\RegionMarkerTok}[1]{{#1}}
    \newcommand{\ErrorTok}[1]{\textcolor[rgb]{1.00,0.00,0.00}{\textbf{{#1}}}}
    \newcommand{\NormalTok}[1]{{#1}}
    
    % Additional commands for more recent versions of Pandoc
    \newcommand{\ConstantTok}[1]{\textcolor[rgb]{0.53,0.00,0.00}{{#1}}}
    \newcommand{\SpecialCharTok}[1]{\textcolor[rgb]{0.25,0.44,0.63}{{#1}}}
    \newcommand{\VerbatimStringTok}[1]{\textcolor[rgb]{0.25,0.44,0.63}{{#1}}}
    \newcommand{\SpecialStringTok}[1]{\textcolor[rgb]{0.73,0.40,0.53}{{#1}}}
    \newcommand{\ImportTok}[1]{{#1}}
    \newcommand{\DocumentationTok}[1]{\textcolor[rgb]{0.73,0.13,0.13}{\textit{{#1}}}}
    \newcommand{\AnnotationTok}[1]{\textcolor[rgb]{0.38,0.63,0.69}{\textbf{\textit{{#1}}}}}
    \newcommand{\CommentVarTok}[1]{\textcolor[rgb]{0.38,0.63,0.69}{\textbf{\textit{{#1}}}}}
    \newcommand{\VariableTok}[1]{\textcolor[rgb]{0.10,0.09,0.49}{{#1}}}
    \newcommand{\ControlFlowTok}[1]{\textcolor[rgb]{0.00,0.44,0.13}{\textbf{{#1}}}}
    \newcommand{\OperatorTok}[1]{\textcolor[rgb]{0.40,0.40,0.40}{{#1}}}
    \newcommand{\BuiltInTok}[1]{{#1}}
    \newcommand{\ExtensionTok}[1]{{#1}}
    \newcommand{\PreprocessorTok}[1]{\textcolor[rgb]{0.74,0.48,0.00}{{#1}}}
    \newcommand{\AttributeTok}[1]{\textcolor[rgb]{0.49,0.56,0.16}{{#1}}}
    \newcommand{\InformationTok}[1]{\textcolor[rgb]{0.38,0.63,0.69}{\textbf{\textit{{#1}}}}}
    \newcommand{\WarningTok}[1]{\textcolor[rgb]{0.38,0.63,0.69}{\textbf{\textit{{#1}}}}}
    
    
    % Define a nice break command that doesn't care if a line doesn't already
    % exist.
    \def\br{\hspace*{\fill} \\* }
    % Math Jax compatability definitions
    \def\gt{>}
    \def\lt{<}
    % Document parameters
    \title{Zaman Serilerinde R ile Veri Analizi; Anomalilerin Ayrıştırılması, Tahminleme Yapılması ve İlişkilerin Saptanması}
    
    
    

    % Pygments definitions
    
\makeatletter
\def\PY@reset{\let\PY@it=\relax \let\PY@bf=\relax%
    \let\PY@ul=\relax \let\PY@tc=\relax%
    \let\PY@bc=\relax \let\PY@ff=\relax}
\def\PY@tok#1{\csname PY@tok@#1\endcsname}
\def\PY@toks#1+{\ifx\relax#1\empty\else%
    \PY@tok{#1}\expandafter\PY@toks\fi}
\def\PY@do#1{\PY@bc{\PY@tc{\PY@ul{%
    \PY@it{\PY@bf{\PY@ff{#1}}}}}}}
\def\PY#1#2{\PY@reset\PY@toks#1+\relax+\PY@do{#2}}

\expandafter\def\csname PY@tok@w\endcsname{\def\PY@tc##1{\textcolor[rgb]{0.73,0.73,0.73}{##1}}}
\expandafter\def\csname PY@tok@c\endcsname{\let\PY@it=\textit\def\PY@tc##1{\textcolor[rgb]{0.25,0.50,0.50}{##1}}}
\expandafter\def\csname PY@tok@cp\endcsname{\def\PY@tc##1{\textcolor[rgb]{0.74,0.48,0.00}{##1}}}
\expandafter\def\csname PY@tok@k\endcsname{\let\PY@bf=\textbf\def\PY@tc##1{\textcolor[rgb]{0.00,0.50,0.00}{##1}}}
\expandafter\def\csname PY@tok@kp\endcsname{\def\PY@tc##1{\textcolor[rgb]{0.00,0.50,0.00}{##1}}}
\expandafter\def\csname PY@tok@kt\endcsname{\def\PY@tc##1{\textcolor[rgb]{0.69,0.00,0.25}{##1}}}
\expandafter\def\csname PY@tok@o\endcsname{\def\PY@tc##1{\textcolor[rgb]{0.40,0.40,0.40}{##1}}}
\expandafter\def\csname PY@tok@ow\endcsname{\let\PY@bf=\textbf\def\PY@tc##1{\textcolor[rgb]{0.67,0.13,1.00}{##1}}}
\expandafter\def\csname PY@tok@nb\endcsname{\def\PY@tc##1{\textcolor[rgb]{0.00,0.50,0.00}{##1}}}
\expandafter\def\csname PY@tok@nf\endcsname{\def\PY@tc##1{\textcolor[rgb]{0.00,0.00,1.00}{##1}}}
\expandafter\def\csname PY@tok@nc\endcsname{\let\PY@bf=\textbf\def\PY@tc##1{\textcolor[rgb]{0.00,0.00,1.00}{##1}}}
\expandafter\def\csname PY@tok@nn\endcsname{\let\PY@bf=\textbf\def\PY@tc##1{\textcolor[rgb]{0.00,0.00,1.00}{##1}}}
\expandafter\def\csname PY@tok@ne\endcsname{\let\PY@bf=\textbf\def\PY@tc##1{\textcolor[rgb]{0.82,0.25,0.23}{##1}}}
\expandafter\def\csname PY@tok@nv\endcsname{\def\PY@tc##1{\textcolor[rgb]{0.10,0.09,0.49}{##1}}}
\expandafter\def\csname PY@tok@no\endcsname{\def\PY@tc##1{\textcolor[rgb]{0.53,0.00,0.00}{##1}}}
\expandafter\def\csname PY@tok@nl\endcsname{\def\PY@tc##1{\textcolor[rgb]{0.63,0.63,0.00}{##1}}}
\expandafter\def\csname PY@tok@ni\endcsname{\let\PY@bf=\textbf\def\PY@tc##1{\textcolor[rgb]{0.60,0.60,0.60}{##1}}}
\expandafter\def\csname PY@tok@na\endcsname{\def\PY@tc##1{\textcolor[rgb]{0.49,0.56,0.16}{##1}}}
\expandafter\def\csname PY@tok@nt\endcsname{\let\PY@bf=\textbf\def\PY@tc##1{\textcolor[rgb]{0.00,0.50,0.00}{##1}}}
\expandafter\def\csname PY@tok@nd\endcsname{\def\PY@tc##1{\textcolor[rgb]{0.67,0.13,1.00}{##1}}}
\expandafter\def\csname PY@tok@s\endcsname{\def\PY@tc##1{\textcolor[rgb]{0.73,0.13,0.13}{##1}}}
\expandafter\def\csname PY@tok@sd\endcsname{\let\PY@it=\textit\def\PY@tc##1{\textcolor[rgb]{0.73,0.13,0.13}{##1}}}
\expandafter\def\csname PY@tok@si\endcsname{\let\PY@bf=\textbf\def\PY@tc##1{\textcolor[rgb]{0.73,0.40,0.53}{##1}}}
\expandafter\def\csname PY@tok@se\endcsname{\let\PY@bf=\textbf\def\PY@tc##1{\textcolor[rgb]{0.73,0.40,0.13}{##1}}}
\expandafter\def\csname PY@tok@sr\endcsname{\def\PY@tc##1{\textcolor[rgb]{0.73,0.40,0.53}{##1}}}
\expandafter\def\csname PY@tok@ss\endcsname{\def\PY@tc##1{\textcolor[rgb]{0.10,0.09,0.49}{##1}}}
\expandafter\def\csname PY@tok@sx\endcsname{\def\PY@tc##1{\textcolor[rgb]{0.00,0.50,0.00}{##1}}}
\expandafter\def\csname PY@tok@m\endcsname{\def\PY@tc##1{\textcolor[rgb]{0.40,0.40,0.40}{##1}}}
\expandafter\def\csname PY@tok@gh\endcsname{\let\PY@bf=\textbf\def\PY@tc##1{\textcolor[rgb]{0.00,0.00,0.50}{##1}}}
\expandafter\def\csname PY@tok@gu\endcsname{\let\PY@bf=\textbf\def\PY@tc##1{\textcolor[rgb]{0.50,0.00,0.50}{##1}}}
\expandafter\def\csname PY@tok@gd\endcsname{\def\PY@tc##1{\textcolor[rgb]{0.63,0.00,0.00}{##1}}}
\expandafter\def\csname PY@tok@gi\endcsname{\def\PY@tc##1{\textcolor[rgb]{0.00,0.63,0.00}{##1}}}
\expandafter\def\csname PY@tok@gr\endcsname{\def\PY@tc##1{\textcolor[rgb]{1.00,0.00,0.00}{##1}}}
\expandafter\def\csname PY@tok@ge\endcsname{\let\PY@it=\textit}
\expandafter\def\csname PY@tok@gs\endcsname{\let\PY@bf=\textbf}
\expandafter\def\csname PY@tok@gp\endcsname{\let\PY@bf=\textbf\def\PY@tc##1{\textcolor[rgb]{0.00,0.00,0.50}{##1}}}
\expandafter\def\csname PY@tok@go\endcsname{\def\PY@tc##1{\textcolor[rgb]{0.53,0.53,0.53}{##1}}}
\expandafter\def\csname PY@tok@gt\endcsname{\def\PY@tc##1{\textcolor[rgb]{0.00,0.27,0.87}{##1}}}
\expandafter\def\csname PY@tok@err\endcsname{\def\PY@bc##1{\setlength{\fboxsep}{0pt}\fcolorbox[rgb]{1.00,0.00,0.00}{1,1,1}{\strut ##1}}}
\expandafter\def\csname PY@tok@kc\endcsname{\let\PY@bf=\textbf\def\PY@tc##1{\textcolor[rgb]{0.00,0.50,0.00}{##1}}}
\expandafter\def\csname PY@tok@kd\endcsname{\let\PY@bf=\textbf\def\PY@tc##1{\textcolor[rgb]{0.00,0.50,0.00}{##1}}}
\expandafter\def\csname PY@tok@kn\endcsname{\let\PY@bf=\textbf\def\PY@tc##1{\textcolor[rgb]{0.00,0.50,0.00}{##1}}}
\expandafter\def\csname PY@tok@kr\endcsname{\let\PY@bf=\textbf\def\PY@tc##1{\textcolor[rgb]{0.00,0.50,0.00}{##1}}}
\expandafter\def\csname PY@tok@bp\endcsname{\def\PY@tc##1{\textcolor[rgb]{0.00,0.50,0.00}{##1}}}
\expandafter\def\csname PY@tok@fm\endcsname{\def\PY@tc##1{\textcolor[rgb]{0.00,0.00,1.00}{##1}}}
\expandafter\def\csname PY@tok@vc\endcsname{\def\PY@tc##1{\textcolor[rgb]{0.10,0.09,0.49}{##1}}}
\expandafter\def\csname PY@tok@vg\endcsname{\def\PY@tc##1{\textcolor[rgb]{0.10,0.09,0.49}{##1}}}
\expandafter\def\csname PY@tok@vi\endcsname{\def\PY@tc##1{\textcolor[rgb]{0.10,0.09,0.49}{##1}}}
\expandafter\def\csname PY@tok@vm\endcsname{\def\PY@tc##1{\textcolor[rgb]{0.10,0.09,0.49}{##1}}}
\expandafter\def\csname PY@tok@sa\endcsname{\def\PY@tc##1{\textcolor[rgb]{0.73,0.13,0.13}{##1}}}
\expandafter\def\csname PY@tok@sb\endcsname{\def\PY@tc##1{\textcolor[rgb]{0.73,0.13,0.13}{##1}}}
\expandafter\def\csname PY@tok@sc\endcsname{\def\PY@tc##1{\textcolor[rgb]{0.73,0.13,0.13}{##1}}}
\expandafter\def\csname PY@tok@dl\endcsname{\def\PY@tc##1{\textcolor[rgb]{0.73,0.13,0.13}{##1}}}
\expandafter\def\csname PY@tok@s2\endcsname{\def\PY@tc##1{\textcolor[rgb]{0.73,0.13,0.13}{##1}}}
\expandafter\def\csname PY@tok@sh\endcsname{\def\PY@tc##1{\textcolor[rgb]{0.73,0.13,0.13}{##1}}}
\expandafter\def\csname PY@tok@s1\endcsname{\def\PY@tc##1{\textcolor[rgb]{0.73,0.13,0.13}{##1}}}
\expandafter\def\csname PY@tok@mb\endcsname{\def\PY@tc##1{\textcolor[rgb]{0.40,0.40,0.40}{##1}}}
\expandafter\def\csname PY@tok@mf\endcsname{\def\PY@tc##1{\textcolor[rgb]{0.40,0.40,0.40}{##1}}}
\expandafter\def\csname PY@tok@mh\endcsname{\def\PY@tc##1{\textcolor[rgb]{0.40,0.40,0.40}{##1}}}
\expandafter\def\csname PY@tok@mi\endcsname{\def\PY@tc##1{\textcolor[rgb]{0.40,0.40,0.40}{##1}}}
\expandafter\def\csname PY@tok@il\endcsname{\def\PY@tc##1{\textcolor[rgb]{0.40,0.40,0.40}{##1}}}
\expandafter\def\csname PY@tok@mo\endcsname{\def\PY@tc##1{\textcolor[rgb]{0.40,0.40,0.40}{##1}}}
\expandafter\def\csname PY@tok@ch\endcsname{\let\PY@it=\textit\def\PY@tc##1{\textcolor[rgb]{0.25,0.50,0.50}{##1}}}
\expandafter\def\csname PY@tok@cm\endcsname{\let\PY@it=\textit\def\PY@tc##1{\textcolor[rgb]{0.25,0.50,0.50}{##1}}}
\expandafter\def\csname PY@tok@cpf\endcsname{\let\PY@it=\textit\def\PY@tc##1{\textcolor[rgb]{0.25,0.50,0.50}{##1}}}
\expandafter\def\csname PY@tok@c1\endcsname{\let\PY@it=\textit\def\PY@tc##1{\textcolor[rgb]{0.25,0.50,0.50}{##1}}}
\expandafter\def\csname PY@tok@cs\endcsname{\let\PY@it=\textit\def\PY@tc##1{\textcolor[rgb]{0.25,0.50,0.50}{##1}}}

\def\PYZbs{\char`\\}
\def\PYZus{\char`\_}
\def\PYZob{\char`\{}
\def\PYZcb{\char`\}}
\def\PYZca{\char`\^}
\def\PYZam{\char`\&}
\def\PYZlt{\char`\<}
\def\PYZgt{\char`\>}
\def\PYZsh{\char`\#}
\def\PYZpc{\char`\%}
\def\PYZdl{\char`\$}
\def\PYZhy{\char`\-}
\def\PYZsq{\char`\'}
\def\PYZdq{\char`\"}
\def\PYZti{\char`\~}
% for compatibility with earlier versions
\def\PYZat{@}
\def\PYZlb{[}
\def\PYZrb{]}
\makeatother


    % Exact colors from NB
    \definecolor{incolor}{rgb}{0.0, 0.0, 0.5}
    \definecolor{outcolor}{rgb}{0.545, 0.0, 0.0}



    
    % Prevent overflowing lines due to hard-to-break entities
    \sloppy 
    % Setup hyperref package
    \hypersetup{
      breaklinks=true,  % so long urls are correctly broken across lines
      colorlinks=true,
      urlcolor=urlcolor,
      linkcolor=linkcolor,
      citecolor=citecolor,
      }
    % Slightly bigger margins than the latex defaults
    
    \geometry{verbose,tmargin=1in,bmargin=1in,lmargin=1in,rmargin=1in}
    
    

    \begin{document}
       \maketitle

\begin{center}
  \author{Büşra Uysal 090130357}

\end{center}
    

  
    \begin{center}
    \includegraphics{logo.jpg}
    \end{center}






\begin{center}
\textbf{Tez Danışmanı: Doç. Dr. Atabey KAYGUN}
\end{center}
        \pagebreak
\tableofcontents
\pagebreak

    

    
    \section{Zaman Serisi Analizi}\label{zaman-serisi-analizi}

Bir zaman serisi, her biri belirli bir \(t\) zamanında kaydedilmekte
olan \(X_t\) gözlemleri kümesidir. Kesikli zaman serileri gözlemlerin
yapıldığı zamanların \(t\) değerlerinin birbirlerinden ayrık olduğunu
söylemektedir. Gözlemler sabit zaman aralıklarında yapıldığı durumlar
kesikli zaman serilerine örnek olarak verilebilir. Sürekli zaman
serileri ise gözlemlerin sürekli bir \(T[0,1]\) aralığında yapılmasıyla
oluşmaktadır.{[}2{]}

Bir çok alanda bilim, mühendislik ve ticaret dalında zaman içinde sıralı
olarak ölçülen veriler bulunmaktadır. Örneğin bankalar her gün faiz
oranlarını ve döviz kurlarını tutar ya da meteoroloji ofisleri sıcaklık
değerlerinin günlük olarak tutarlar. Bir değişken, zaman içinde belirli
sabit bir aralıkta ölçüldüğünde, elde edilen veriler bir zaman serisi
oluşturmaktadır.{[}1{]}

Ayrıca zaman serileri bir çok alanda karşımıza çıkan belirli aralıklarla
ölçümlenmiş veri kümeleri olarak da tanımlanabilir. Temel bir veri
analizinde birbirine benzer ve bağımsız dağılmış veriler mevcut iken
zaman serilerinde birbiriyle ilişkili veriler bulunur. Zaman
serilerindeki analizin amacı; geçmiş verilerdeki anomalileri saptamak,
birbirleriyle ilişkilerini gözlemlemek ve gelecek için tahminleme
yapabilmektir. Basit tanımlayıcı analizlerle verinin anlaşılması
sağlanırken, kapsamlı bir analiz ile gözlenen verilerin rassal
modellemesi yapılabilmektedir.{[}3{]}

Zaman serilerinin temel özellikleri trend ve mevsimsel değişimlerdir.
Bunlar matematiksel fonksiyonları ile deterministik olarak
modellenebilirler. Ancak, zaman serilerinin bir diğer önemli özelliği
ise birbirleriyle ilişkili olmaları yani birbirlerine yakın gözlemlerin
korelasyon içermesidir. Bir zaman dizisi analizindeki temel amaç, bu
istatistiksel ilişkiyi ve verideki temel özellikleri uygun istatistiksel
modeller ve tanımlayıcı yöntemler kullanarak açıklamaktır. Yöntemlerin
uygunluğu ise daha sonrasında uygulanacak istatiksel testlerle
ölçümlenebilmektedir. {[}1{]}

\subsection{Zaman Serisi Analizindeki
Hedefler}\label{zaman-serisi-analizindeki-hedefler}

\subsubsection{Keşif Analizi:}\label{keux15fif-analizi}

Zaman serileri keşif analizleri ağırlıklı olarak zaman serilerinin temel
özelliklerinin (mevsimsellik, trend, korelasyon...) saptanması, grafik
çizimlerinin yapılması yani deterministik ve stokastik kısımların doğru
şekilde incelenebilmesi için yapılır.

\subsubsection{Modelleme:}\label{modelleme}

Zaman serisine uygun bir modellenmenin yapılması keşif analizinde elde
edilen bilgiler yardımıyla yapılmaktadır. Düzgün modelleme yapılmayan
analizler bir sonraki adımda doğru sonuçlar vermeyecektir.

\subsubsection{Tahmin:}\label{tahmin}

Zaman serileri analizinde sıkça kullanılan yapılardan biri olan
gelecekteki gözlemlerin tahmin edilmesidir. Fakat gelecekteki verilerin
tahmini zaman serisinin geçmiş ve şimdiki özelliklerinin devam ettiği
varsayımına dayanır. İyi bir tahmin yapılabilmesi için keşif analizinin
doğru yapılması ve modelin doğru kurulması gerekmektedir.

\subsubsection{Regresyon:}\label{regresyon}

Zaman serisi analizinde gözlemlerin gelecekteki tahminlerini yapmaktan
başka sıkça kullanılan diğer bir yapı ise gözlemler arasındaki ilişkinin
saptanmasıdır. Bu şekilde zaman serileri daha açıklayıcı hale
gelmektedir.

\subsubsection{Süreç Kontrol:}\label{suxfcreuxe7-kontrol}

Optimal yönetim ve kalite kontrol amacıyla birçok üretim veya diğer
süreçler ölçümlenir. Bu genellikle rassal bir modelin uygun olduğu zaman
serisi verisi ile sonuçlanır. Bu, verilerdeki sinyalin anlaşılmasını
sağlamaktadır. Üretimdeki hangi dalgalanmaların normal olduğunu ve
hangilerinin müdahale gerektirdiğini izlemek mümkün hale gelir.{[}3{]}

\subsection{Zaman Serisi Analizindeki Anomalilerin
tespiti}\label{zaman-serisi-analizindeki-anomalilerin-tespiti}

Anomaliler, iyi tanımlanmış normal davranış kavramına uymayan
verilerdeki yapılardır. Bu yapıları bulma problemi anomallilerin tespiti
olarak adlandırılır. Anomali tespitinin önemi, verideki anormalliklerin
çok çeşitli uygulama alanlarında önemli ve eyleme geçirilebilir
bilgilere dönüşebilir olmasıdır.

ARIMA, anomali tespiti için kullanılan bir modeldir. Sinyalleri doğru
tahmin etmek ve anormallikleri bulmak için yeterince güçlü bir model
olan ARIMA modelinin uygulanması için verinin trend, mevsimsellik gibi
özelliklerinden arındırılması gerekmektedir.{[}6{]}

\subsection{Veri setinin yüklenmesi}\label{veri-setinin-yuxfcklenmesi}

Veri analizinde kullanacağımız veri seti Londra'daki platin fiyatlarının
dolar bazında 1990 ile 2018 yılları arasındaki günlük sabit
değerleridir. Sabit fiyat, dünya çapındaki müşterilere ait siparişlerin
eşleştirilmesini temsil eder.

Platin çok değerli madenler arasındadır. Çok fazla değerli olmasının
nedeni ender bulunuşu ve kullanım alanlarının fazlasıyla geniş
olmasıdır. Otomotiv, diş hekimliği, jet ve füze motorları, laboratuvar
gibi birçok alanda kullanılmaktadır.

\subsection{Zaman serilerinin
açıklanması}\label{zaman-serilerinin-auxe7ux131klanmasux131}

Zaman serisi analizinde yapılması gereken ilk aşama verinin
yüklenmesidir. Analizde kullanılacak paketlerin yüklenmesinin ardından,
gerekli kütüphanelerde çağırılmalıdır. Bunun için aşağıdaki R kodları
kullanılmaktadır.

    \begin{Verbatim}[commandchars=\\\{\}]
{\color{incolor}In [{\color{incolor}2}]:} \PY{c+c1}{\PYZsh{}install.packages(\PYZdq{}Quandl\PYZdq{}, repos=\PYZsq{}http://cran.us.r\PYZhy{}project.org\PYZsq{})}
        \PY{c+c1}{\PYZsh{}install.packages(\PYZdq{}corrplot\PYZdq{}, repos=\PYZsq{}http://cran.us.r\PYZhy{}project.org\PYZsq{})}
        \PY{k+kn}{library}\PY{p}{(}Quandl\PY{p}{)}
        \PY{k+kn}{library}\PY{p}{(}corrplot\PY{p}{)}
        \PY{k+kn}{library}\PY{p}{(}forecast\PY{p}{)}
        \PY{k+kn}{library}\PY{p}{(}tseries\PY{p}{)}
        PlatinumPrices\PY{o}{\PYZlt{}\PYZhy{}}Quandl\PY{p}{(}\PY{l+s}{\PYZdq{}}\PY{l+s}{LPPM/PALL\PYZdq{}}\PY{p}{)}
\end{Verbatim}


    Verinin yüklenmesi tamamlandıktan sonra verinin genel bir çerçevede
incelenmesi yapılması gereken aşamalardan birisidir. Bunun için özetine
bakılması, grafiğinin çizilmesi gerekir.
\pagebreak

    \begin{Verbatim}[commandchars=\\\{\}]
{\color{incolor}In [{\color{incolor}3}]:} \PY{k+kp}{summary}\PY{p}{(}PlatinumPrices\PY{p}{)}
\end{Verbatim}


    
    \begin{verbatim}
      Date                USD AM            EUR AM           GBP AM     
 Min.   :1990-04-02   Min.   :  78.75   Min.   : 116.1   Min.   : 40.4  
 1st Qu.:1997-04-11   1st Qu.: 161.56   1st Qu.: 246.3   1st Qu.: 99.7  
 Median :2004-04-22   Median : 325.00   Median : 421.1   Median :179.4  
 Mean   :2004-04-21   Mean   : 396.02   Mean   : 433.2   Mean   :254.6  
 3rd Qu.:2011-05-04   3rd Qu.: 640.00   3rd Qu.: 577.1   3rd Qu.:414.2  
 Max.   :2018-05-17   Max.   :1128.00   Max.   :1179.5   Max.   :819.5  
                                        NA's   :2213                    
     USD PM            EUR PM           GBP PM     
 Min.   :  78.25   Min.   : 122.8   Min.   : 41.0  
 1st Qu.: 161.75   1st Qu.: 246.5   1st Qu.: 99.6  
 Median : 325.00   Median : 419.8   Median :179.7  
 Mean   : 396.25   Mean   : 433.6   Mean   :254.8  
 3rd Qu.: 640.00   3rd Qu.: 578.1   3rd Qu.:415.1  
 Max.   :1129.00   Max.   :1179.7   Max.   :819.2  
 NA's   :53        NA's   :2248     NA's   :53     
    \end{verbatim}

    
    Yukarıdaki veri setimizdeki tüm değerlerin özeti görülmektedir.
Çıkarılan özete göre, bu veri seti 1990 ile 2018 arasındaki platin
fiyatlarının açılış ve kapanış fiyatlarını dolar, euro gibi para
birimlerini baz alarak göstermektedir. Fakat yapılacak ilk aşama boyunca
platinin dolar açılış fiyatı kullanıcak ve bu veriler arasında anomali
tespiti yapılacaktır. İkinci sütunda dolar açılış fiyatlarının minimum
fiyat 78.75(dolar) iken maximum 1128 (dolar) olduğu görülmektedir.
Ayrıca yukarıdaki özet PlatinumPrices nesnesinin içerisinde yer alan tüm
para birimleri bazında platin fiyatları içermektedir.

Zaman serileri ts sınıfının bir R nesnesi olarak saklanır. Zaman
serisinin doğru analizi için PlatinumPrices nesnesinin ikinci sütununda
yer alan dolar açılış fiyatları zaman serisine dönüştürülmelidir. Zaman
serisi nesneleri, yukarıda verilen işlevlerin başlangıcı, sonu ve
sıklığını içeren bir dizi yönteme sahiptir. Fakat zaman serileri
oluşturulurken sadece frekansının yazılması da yeterli olacaktır.

Ayrıca zaman serisi analizindeki en önemli adımlardan biri verileri
çizmektir; yani, grafiği oluşturmaktır. Bu işlemler aşağıdaki R
komutlarıyla yapılabilir.

    \begin{Verbatim}[commandchars=\\\{\}]
{\color{incolor}In [{\color{incolor}4}]:} PlatinumPricests\PY{o}{\PYZlt{}\PYZhy{}}ts\PY{p}{(}PlatinumPrices\PY{p}{[}\PY{p}{,}\PY{l+m}{2}\PY{p}{]}\PY{p}{,}
                             freq\PY{o}{=}\PY{l+m}{365}\PY{p}{)}
        plot\PY{p}{(}PlatinumPricests\PY{p}{,}type\PY{o}{=}\PY{l+s}{\PYZsq{}}\PY{l+s}{s\PYZsq{}}\PY{p}{,}
             xlab\PY{o}{=}\PY{l+s}{\PYZdq{}}\PY{l+s}{Time\PYZdq{}}\PY{p}{,}
             ylab\PY{o}{=}\PY{l+s}{\PYZdq{}}\PY{l+s}{(\PYZdl{})\PYZdq{}}\PY{p}{,} 
             main\PY{o}{=}\PY{l+s}{\PYZdq{}}\PY{l+s}{Platinum Prices 1990\PYZhy{}2018\PYZdq{}}\PY{p}{)}
\end{Verbatim}


    \begin{center}
    \adjustimage{max size={0.9\linewidth}{0.9\paperheight}}{output_5_0.png}
    \end{center}
    { \hspace*{\fill} \\}
    
    Yukarıda çizilen grafikten görüldüğü üzere ele alacağımız veri seti
1990-2018 yılları arasında Platin'in dolar fiyatını içermektedir. Veri
setinin grafiğini çizmek için yukarıdaki R komutu kullanılır. 2008
yılına kadar neredeyse durağan seyreden fiyatlar 2008-2010 yılları
arasında bir dalgalanma olduğu çizilen grafikte görülmektedir. 2010-2015
yıllarında yüksek fiyatlarda seyretmiş ve daha sonra azalarak
günümüzdeki fiyatına ulaşmaktadır.Bu dalgalanmayı içeren sinyaller veri
analizi aşamalarında saptanacak ve veriden uzaklaştıralacaktır.
Mevsimsellik; aylar, haftanın günleri, mevsimler, vb. gibi
sürelerleilgili düzenli ve yinelenen yüksek ve düşük örüntüleri ifade
etmektedir. Yukarıda görüldüğü üzere veride herhangi bir mevsimselliğe
rastlanmamaktadır. Trend; ortalama olarak, ölçümlerin zamanla artma
(veya azaltma) eğiliminde olması anlamına gelir. Yıllar içerisinde
platin fiyatlarında bir artış olduğu görülmektedir. Aykırı değerler;
regresyon çizgisinden çok uzak değerler olarak düşünülebilir.

    \subsection{Zaman serilerinin ayrıştırılması
(Decomposition)}\label{zaman-serilerinin-ayrux131ux15ftux131rux131lmasux131-decomposition}

Zaman serilerinin çoğu trend ve mevsimsel etkiye sahiptir.

\(Trend:\) ortalama olarak, ölçümlerin zamanla artma (veya azaltma)
eğiliminde olması anlamına gelir.

\(Mevsimsellik:\)yani mevsimler,aylar, haftanın günleri, gibi sürelerle
ilgili düzenli ve yinelenen yüksek ve düşük örüntülerdir. Serilerin
seviyesinde veya varyansta ani değişiklikler var mı?

Ayrıştırma (decomposition) modelleri ile serilerin bu özelliklerden
ayrılması sağlanır. Basit bir toplamsal(additive) model aşağıdaki
şekilde gösterilebilir. \[X_t=m_t+s_t+R_t\] Buradaki \(X_t\) zaman
serisini gösterir.\(m_t\) trend bileşenini ifade ederken, \(s_t\)
mevsimsel etkiyi göstermektedir. \(R_t\) genellikle ortalaması sıfır
olan korelasyonlu rastgele değişkenlerin bir dizisi yani kalan terimi
göstermektedir.Amaç, Rt'nin sabit bir zaman serileri süreci olduğu bir
ayrıştırma bulmaktır. Böyle bir model hava yolcu rezervasyonları,
işsizlik gibi bir çok modelde karşımıza çıkabilir. Ayrıca tüm bu
serilerin daha yakından incelenmesi, mevsimsel etkinin ve trendin
arttığında rasgele değişimin arttığını göstermektedir. Bu gibi
durumlarda, çarpımsal(multiplicative) model kullanılmaktadır. Çarpımsal
model aşağıdaki şekilde ifade edilebilir.

\[X_t=m_t*s_t*R_t\]

Trend \(m_t\) , mevsimsellik \(s_t\) ve kalan terimlerin \(R_t\)
tahminlemesi bir çok açıdan yapılabilir. Örneğin Decompose() fonksiyonu
trend, mevsimsellik değerlerini modelin toplamsal ya da çarpımsal
parametresine göre ayrıştırmaktadır. Hem toplamsal(additive) hem de
çarpımsal(multiplicative) model için zaman serisinin ayrıştırılması
gösterilmiştir. R'nin stl () komutu, periyodik bir zaman serisinin
trend, mevsimsellik ve geri kalanına ayrışmasını sağlayan başka bir
yöntemdir. Tüm tahminler LOESS yumuşaklığına dayanmaktadır. Çıktısı
neredeyse decompose () ile elde ettiğimiz değerlere benzer olmasına
rağmen, bu fonksiyon daha güvenilir sonuçlar verebilmektedir. {[}3{]}

\subsection{Beyaz Gürültü}\label{beyaz-guxfcruxfcltuxfc}

Veri setinin içindeki trend, mevsimsellik ve üssel dağılımları yukarıda
anlatılan işlemlerle çıkarıldıktan sonra elimizde kalan grafiğin eğer
herhangi bir aykırı değerler içermiyorsa beyaz gürültü olması
beklenmektedir. Trendin ayrıştırılması için linear regresyon,
mevsimselliğin ayrıştırılması için fourrier serileri ve üssel dağılımın
ayrıştırılması için ise ARIMA modellerinin çıkarılması gerekmektedir.
Aslında decompose() ve stl() fonksiyonları bu trend ve mevsimsellik için
bu ayrıştırmayı yapmaktadır. Bu yüzden ayrışmadan elde kalan \(R_t\)
random fonksiyonunu gürültü fonksiyonuna yakın olması beklenmektedir.

\paragraph{Beyaz gürültü :}\label{beyaz-guxfcruxfcltuxfc-1}
 Bir \({X_t}\) fonksiyonu bağımsız ve ortalaması sıfır ve varyansı
\(σ^2\) olarak dağılıma sahipse Beyaz Gürültü olarak adlandırılır. Bir
beyaz gürültü aşağıdaki notasyon ifade edilebilir. {[}2{]}
\[{Xt} ∼ N(0, σ^2)\]

Aşağıda decompose() işleminin hem toplamsal, hem çarpımsal modele göre
ayrıştırılması aynı zamanda stl() e göre yapılan ayrıştırmanın R
komutları görülmektedir. Ayrıca bu ayrıştırmaların grafikleri
oluşturulmuştur.

    \begin{Verbatim}[commandchars=\\\{\}]
{\color{incolor}In [{\color{incolor}5}]:} PlatinumPricesDecomposeA\PY{o}{\PYZlt{}\PYZhy{}}decompose\PY{p}{(}PlatinumPricests\PY{p}{,} 
                                            type \PY{o}{=} \PY{k+kt}{c}\PY{p}{(}\PY{l+s}{\PYZdq{}}\PY{l+s}{additive\PYZdq{}}\PY{p}{)}\PY{p}{,} 
                                            filter \PY{o}{=} \PY{k+kc}{NULL}\PY{p}{)}
        plot\PY{p}{(}PlatinumPricesDecomposeA\PY{p}{)}
        PlatinumPricesDecomposeM\PY{o}{\PYZlt{}\PYZhy{}}decompose\PY{p}{(}PlatinumPricests\PY{p}{,} 
                                            type \PY{o}{=} \PY{k+kt}{c}\PY{p}{(}\PY{l+s}{\PYZdq{}}\PY{l+s}{multiplicative\PYZdq{}}\PY{p}{)}\PY{p}{,} 
                                            filter \PY{o}{=} \PY{k+kc}{NULL}\PY{p}{)}
        plot\PY{p}{(}PlatinumPricesDecomposeM\PY{p}{)}
        PlatinumPricesSTL\PY{o}{\PYZlt{}\PYZhy{}}stl\PY{p}{(}PlatinumPricests\PY{p}{,}s.window\PY{o}{=}\PY{l+s}{\PYZsq{}}\PY{l+s}{periodic\PYZsq{}}\PY{p}{)}
        plot\PY{p}{(}PlatinumPricesSTL\PY{p}{,}main\PY{o}{=}\PY{l+s}{\PYZsq{}}\PY{l+s}{Decomposition of STL\PYZsq{}}\PY{p}{)}
\end{Verbatim}


    \begin{center}
    \adjustimage{max size={0.9\linewidth}{0.9\paperheight}}{output_8_0.png}
    \end{center}
    { \hspace*{\fill} \\}
    
    \begin{center}
    \adjustimage{max size={0.9\linewidth}{0.9\paperheight}}{output_8_1.png}
    \end{center}
    { \hspace*{\fill} \\}
    
    \begin{center}
    \adjustimage{max size={0.9\linewidth}{0.9\paperheight}}{output_8_2.png}
    \end{center}
    { \hspace*{\fill} \\}
    
    Zaman serisi ayrıştırmasındaki \(observed\) ile gösterilen verilerin
gerçek grafiğidir. \(Trend\) bileşeni \(m_t\) yani eğilimi gösterirken,
\(seasonal\) verideki mevsimselliği yani \(s_t\) temsil etmektedir.
\(Random\) ise gerçek verilerden \(m_t\) ve \(s_t\) değerlerinin
çıkarılması ile oluşur. Bu da \(R_t\) kalan değeri gösterir. Stl() ile
yapılan ayrıştırma işleminde ise \(remainder\) trend ve mevsimsellik
çıkarıldığındaki kalan değeri ifade etmektedir.

decompose() ve stl() ile yapılan ayrıştırmalar incelendiğinde hangisinin
verinin ayrıştırmada doğru sonuçlar vereceği yani random
fonksiyonlarından hangisinin beyaz gürültü fonksiyonuna daha yakın
olduğu göz ile tespit edilememektedir. Bunun için normallik testlerinden
yararlanılması gerekmektedir.

\subsubsection{Beyaz Gürültü
Testleri}\label{beyaz-guxfcruxfcltuxfc-testleri}

Yukarıda yapılan ayrıştırma işlemlerinin amacı belirgin bir sapma
göstermeyen ve özellikle belirgin bir eğilim veya mevsimselliğe sahip
olmayan bir seri üretmektir.Bu aşamadan bir sonraki adım tahmini gürültü
serisini modellemektir (Yani, verilerden trend ve mevsimsel bileşenleri
tahmin ederek ve çıkararak elde edilen artıklar).Eğer bu artıklar
arasında bir bağımlılık yoksa, onları bağımsız rasgele değişkenlerin
gözlemleri olarak görebiliriz ve onların ortalama ve varyanslarını
tahmin etmek dışında başka bir modelleme yoktur. Ancak, artıklar
arasında önemli bir bağımlılık varsa, o zaman bağımlılığı açıklayan
gürültü için daha karmaşık bir durağan zaman serisi modelini aramamız
gerekir.

\(R_t\) serisinin beyaz gürültü olup olmadığı için testler
kullanılmalıdır. Eğer beyaz gürültü ise zaman serisi açıklanmış
olacaktır ve içerisinde trend ve mevsimsellikten başka bir anamoliye
rastlanmamıştır diyebiliriz.Fakat değil ise o zaman ARIMA gibi
modellerle içerisindeki anomaliler saptanıcaktır.{[}2{]}

\subsubsection{Shapiro-Wilks test}\label{shapiro-wilks-test}

Shapiro ve Wilk tarafından geliştirilen Shapiro-Wilk testi, çoğu durumda
en güçlü ve çok amaçlı testtir. Son yıllarda, SW testi, çok çeşitli
alternatif testlere kıyasla iyi güç özellikleri nedeniyle normalliğin
tercih edilen testi haline gelmiştir.W test istatistiği dağılımın normal
olup olmadığına karar verilmesi için kullanılır. W test istatistiği
aşağıdaki notasyon ile ifade edilir.

\[W=\frac{\sum_i(a_ix_i)^2}{\sum_i(x_i-\bar{x})^2}\]

\(x_i\) veri setindeki değerleri, \(\bar{x}\),ortalamayı ve n,
gözlemlerin sayısını temsil eder.\({x_t}\) herhangi bir veri seti olmak
üzere normalliğinin testi için kullanılan Shapiro-Wil test için
Shapiro.test(\(x_t\)) R komutu kullanılır. Eğer p değeri, istenilen
\(\alpha\) değerinden küçük ise o zaman normallik hipotezini reddedilir.
Test istatistiği 0 \textless{}W ≤ 1 arasındadır. W değerinin 1'e yakın
değerler için normallik hipotezi reddedilmez. Daha küçük W için
reddedilecektir. Yani W değeri büyük olan \(R_t\) beyaz gürültüye daha
çok benzemektedir. Test sadece n ≥ 3 değerleri için geçerlidir ve R
uygulaması n 5,000'e kadar izin vermektedir.

\subsubsection{Kolmogorov-Smirnov test}\label{kolmogorov-smirnov-test}

KS testi ilk olarak Kolmogorov tarafından önerilmiştir ve daha sonra
Smirnov tarafından geliştirilmiştir. Bu test verilerin kümülatif
dağılımını beklenen kümülatif normal dağılım ile karşılaştırır. Yani iki
örneklemi karşılaştırmak için kullanılmaktadır.Test istatistiği D
normalliğe karar vermek için kullanılır. D test istatistiği için küçük
değerleri, normalliği ifade etmektedir. D test istatistiği aşağıdaki
notasyon ile ifade edilir.

\[ D=\max[F_x(u)-F_y(u)]\]

Fakat bizim test etmemiz gereken bir örneklem olduğu için R'da
kullanılan ks.test() fonskiyonunda ikinci örneklem için 'pnorm' yani
örnek bir kümülatif dağılımı temsil etmektedir. {[}3{]}

Shapiro-Wilks test ile 3 ile 5000 tane veri test edilebilirken
Kolmogorov-Smirnov testi ile daha çok veri test edilebilir. Aşağıda
ayrıştırmalardan kalan veri setleri için normallik testleri
uygulanmıştır. Shapiro-Wilk testi 5000 veri ile çalıştığı için mevcut
veri setimizin içerisinden örneklem oluşturarak daha doğru sonuçlar elde
edilecektir.

    \begin{Verbatim}[commandchars=\\\{\}]
{\color{incolor}In [{\color{incolor}6}]:} SamplePlatinumPricesDecomposeA\PY{o}{\PYZlt{}\PYZhy{}}\PY{k+kp}{sample}\PY{p}{(}PlatinumPricesDecomposeA\PY{o}{\PYZdl{}}random\PY{p}{,}\PY{l+m}{5000}\PY{p}{)}
        SamplePlatinumPricesDecomposeM\PY{o}{\PYZlt{}\PYZhy{}}\PY{k+kp}{sample}\PY{p}{(}PlatinumPricesDecomposeM\PY{o}{\PYZdl{}}random\PY{p}{,}\PY{l+m}{5000}\PY{p}{)}
        SamplePlatinumPricesDecomposeSTL\PY{o}{\PYZlt{}\PYZhy{}}\PY{k+kp}{sample}\PY{p}{(}PlatinumPricesSTL\PY{o}{\PYZdl{}}time.series\PY{p}{[}\PY{p}{,}\PY{l+m}{3}\PY{p}{]}\PY{p}{,}\PY{l+m}{5000}\PY{p}{)}
        shapiro.test\PY{p}{(}SamplePlatinumPricesDecomposeA\PY{p}{)}
        shapiro.test\PY{p}{(}SamplePlatinumPricesDecomposeM\PY{p}{)}
        shapiro.test\PY{p}{(}SamplePlatinumPricesDecomposeSTL\PY{p}{)}
        ks.test\PY{p}{(}PlatinumPricesDecomposeA\PY{o}{\PYZdl{}}random\PY{p}{,}
                \PY{l+s}{\PYZsq{}}\PY{l+s}{pnorm\PYZsq{}}\PY{p}{,}
                mean\PY{o}{=}\PY{l+m}{0}\PY{p}{,}
                sd\PY{o}{=}\PY{k+kp}{sqrt}\PY{p}{(}var\PY{p}{(}na.omit\PY{p}{(}PlatinumPricesDecomposeA\PY{o}{\PYZdl{}}random\PY{p}{)}\PY{p}{)}\PY{p}{)}\PY{p}{)}
        ks.test\PY{p}{(}PlatinumPricesDecomposeM\PY{o}{\PYZdl{}}random\PY{p}{,}
                \PY{l+s}{\PYZsq{}}\PY{l+s}{pnorm\PYZsq{}}\PY{p}{,}
                mean\PY{o}{=}\PY{l+m}{0}\PY{p}{,}
                sd\PY{o}{=}\PY{k+kp}{sqrt}\PY{p}{(}var\PY{p}{(}na.omit\PY{p}{(}PlatinumPricesDecomposeM\PY{o}{\PYZdl{}}random\PY{p}{)}\PY{p}{)}\PY{p}{)}\PY{p}{)}
        ks.test\PY{p}{(}PlatinumPricesSTL\PY{o}{\PYZdl{}}time.series\PY{p}{[}\PY{p}{,}\PY{l+m}{3}\PY{p}{]}\PY{p}{,}
                \PY{l+s}{\PYZsq{}}\PY{l+s}{pnorm\PYZsq{}}\PY{p}{,}
                mean\PY{o}{=}\PY{l+m}{0}\PY{p}{,}
                sd\PY{o}{=}\PY{k+kp}{sqrt}\PY{p}{(}var\PY{p}{(}na.omit\PY{p}{(}PlatinumPricesSTL\PY{o}{\PYZdl{}}time.series\PY{p}{[}\PY{p}{,}\PY{l+m}{3}\PY{p}{]}\PY{p}{)}\PY{p}{)}\PY{p}{)}\PY{p}{)}
\end{Verbatim}


    
    \begin{verbatim}

	Shapiro-Wilk normality test

data:  SamplePlatinumPricesDecomposeA
W = 0.88658, p-value < 2.2e-16

    \end{verbatim}

    
    
    \begin{verbatim}

	Shapiro-Wilk normality test

data:  SamplePlatinumPricesDecomposeM
W = 0.96362, p-value < 2.2e-16

    \end{verbatim}

    
    
    \begin{verbatim}

	Shapiro-Wilk normality test

data:  SamplePlatinumPricesDecomposeSTL
W = 0.88739, p-value < 2.2e-16

    \end{verbatim}

    
    
    \begin{verbatim}

	One-sample Kolmogorov-Smirnov test

data:  PlatinumPricesDecomposeA$random
D = 0.10989, p-value < 2.2e-16
alternative hypothesis: two-sided

    \end{verbatim}

    
    
    \begin{verbatim}

	One-sample Kolmogorov-Smirnov test

data:  PlatinumPricesDecomposeM$random
D = 1, p-value < 2.2e-16
alternative hypothesis: two-sided

    \end{verbatim}

    
    
    \begin{verbatim}

	One-sample Kolmogorov-Smirnov test

data:  PlatinumPricesSTL$time.series[, 3]
D = 0.11444, p-value < 2.2e-16
alternative hypothesis: two-sided

    \end{verbatim}

    
    Shapiro-Wilk testinin çıktılarına göre en büyük W değeri çarpımsal model
kullanılarak yapılan ayrıştırmanın 5000 verilik örnekleminden kalan
verilerinde görülmektedir. Yani shapiro-wilk testine göre çarpımsal
modelin \(R_t\) değeri normalliğe daha yakındır. Fakat
Kolmogorov-Smirnov testi ise D değeri küçük olan toplamsal ve stl ile
yapılan ayrıştırmadan kalan \(R_t\)'nin normalliğe daha yakın olduğunu
göstermektedir. Buradan anlaşılacağı üzere herhangi bir ayrıştırmanın en
iyi olacağı sonucuna varılamamıştır. Yani shapiro-wilk testi ile
Kolmogorov-Smirnov testi birbirleriyle çelişki çıktılara sahiptir.
Bundan dolayı bir sonraki adımda yapılacak olan Box-Jenkins modelleri
hem toplamsal hem de çarpımsal modellerin kalanları için uygulanmalıdır.
Uygun Box-Jenkins modelin tespiti için serinin durağan olup olmadığına
karar verilmesi gerekir. Durağan bir seri için Otoregresyon Modelleri
(AR),Hareketli Ortalama Modelleri (MA),Otoregresif Hareketli Ortalama
Modelleri(ARMA) kullanılırken durağan olmayan seriler için bütünleşik
hareketli otoregresif modeller (ARIMA) kullanılır. Modellerin
uygulanması için öncelikli olarak serinin durağan olup olmadığının
testleri yapılmalı ve durağan olması sağlanmalıdır.

\subsection{Durağanlık}\label{duraux11fanlux131k}

Bir zaman serisi \({X_t, t=0, ± 1, ...}\), eğer her h tam sayısı için
\({X_{t+h}, t=0, ± 1, ...}\) zaman kaydırılmış serisi ile benzer
istatistiksel özelliklere sahipse durağandır. \{\(X_t\)\} bir zaman
serisi aşağıdaki özelliklere sahip olsun.

Beklenen Değer \(\varphi\): E(\(X_t\)),

Varyans \(\sigma^2\): Var(\(X_t\)) ,

Kovaryans \(\gamma\): Cov (\(X_t\),\(X_{t+h}\))

Başka bir deyişle durağan zaman serisi verilerinin belirli bir zaman
sürecinde sürekli artma veya azalmanın olmadığı, verilerin zaman boyunca
bir yatay eksen boyunca saçılım gösterdiği biçimde tanımlanır. Genel bir
tanımlama ile, sabit ortalama, sabit varyans ve seriye ait iki değer
arasındaki farkın zamana değil, yalnızca iki zaman değeri arasındaki
farka bağlı olması şeklinde ifade edilir. Bir zaman serisinin durağan
olup olmadığı 2 yöntem ile saptanabilir;

1.Serilerin zaman yolu grafiğinde ve onun korelogramında otokorelasyon
ve kısmi otokorelasyon katsayıları üzerinde yapılan subjektif yargılar
ile

2.Birim köklerin varlığını için istatistiki testlerin kullanılması ile.

Durağanlık için kullanılan bazı testlere örnek olarak Genişletilmiş
Dickey-Fuller (ADF) Testi ve KPSS Testi verilebilir.(Marcel
Dettling,2014)

\subsubsection{Genişletilmiş Dickey-Fuller (ADF)
Testi}\label{geniux15fletilmiux15f-dickey-fuller-adf-testi}

Genişletilmiş Dickey Fuller Testi (ADF) durağanlık için birim kök
testidir. Birim kökler, zaman serisi analizinizde öngörülemeyen
sonuçlara neden olabilir. Artırılmış Dickey-Fuller testi, seri
korelasyon ile kullanılabilir. ADF testi, Dickey-Fuller testinden daha
karmaşık modelleri ele alabilir ve aynı zamanda daha güçlüdür. Sıfır
hipotezi serinin durağan olmadığını ve birim kök içerdiğini, buna karşın
alternatif hipotez ise seride birim kök olmadığını ve durağan olduğu
ifade eder.

\subsubsection{KPSS (Kwiatkowski-Phillips-Schmidt-Shin)
Testi}\label{kpss-kwiatkowski-phillips-schmidt-shin-testi}

KPSS testinde amaç gözlenen serideki deterministik trendi arındırarak
serinin durağan olmasını sağlamaktır. Bu testte kurulan birim kök
hipotezi ADF testinde kurulan hipotezlerden farklıdır. Sıfır hipotezi
serinin durağan olduğunu ve birim kök içermediğini, buna karşın
alternatif hipotez ise seride birim kök olduğunu ve durağan olmadığını
ifade eder. Boş hipotezdeki durağanlık trend durağanlıktır. Çünkü
seriler trendden arındırılmışlardır. {[}1{]}

Aşağıda durağanlık tespiti için ADF ve KPSS testleri yapılmıştır.

    \begin{Verbatim}[commandchars=\\\{\}]
{\color{incolor}In [{\color{incolor}7}]:} adf.test\PY{p}{(}na.omit\PY{p}{(}SamplePlatinumPricesDecomposeA\PY{p}{)}\PY{p}{)}
        adf.test\PY{p}{(}na.omit\PY{p}{(}SamplePlatinumPricesDecomposeM\PY{p}{)}\PY{p}{)}
        adf.test\PY{p}{(}na.omit\PY{p}{(}SamplePlatinumPricesDecomposeSTL\PY{p}{)}\PY{p}{)}
        kpss.test\PY{p}{(}na.omit\PY{p}{(}SamplePlatinumPricesDecomposeA\PY{p}{)}\PY{p}{)}
        kpss.test\PY{p}{(}na.omit\PY{p}{(}SamplePlatinumPricesDecomposeM\PY{p}{)}\PY{p}{)}
        kpss.test\PY{p}{(}na.omit\PY{p}{(}SamplePlatinumPricesDecomposeSTL\PY{p}{)}\PY{p}{)}
\end{Verbatim}


    \begin{Verbatim}[commandchars=\\\{\}]
Warning message in adf.test(na.omit(SamplePlatinumPricesDecomposeA)):
"p-value smaller than printed p-value"
    \end{Verbatim}

    
    \begin{verbatim}

	Augmented Dickey-Fuller Test

data:  na.omit(SamplePlatinumPricesDecomposeA)
Dickey-Fuller = -16.731, Lag order = 16, p-value = 0.01
alternative hypothesis: stationary

    \end{verbatim}

    
    \begin{Verbatim}[commandchars=\\\{\}]
Warning message in adf.test(na.omit(SamplePlatinumPricesDecomposeM)):
"p-value smaller than printed p-value"
    \end{Verbatim}

    
    \begin{verbatim}

	Augmented Dickey-Fuller Test

data:  na.omit(SamplePlatinumPricesDecomposeM)
Dickey-Fuller = -17.946, Lag order = 16, p-value = 0.01
alternative hypothesis: stationary

    \end{verbatim}

    
    \begin{Verbatim}[commandchars=\\\{\}]
Warning message in adf.test(na.omit(SamplePlatinumPricesDecomposeSTL)):
"p-value smaller than printed p-value"
    \end{Verbatim}

    
    \begin{verbatim}

	Augmented Dickey-Fuller Test

data:  na.omit(SamplePlatinumPricesDecomposeSTL)
Dickey-Fuller = -16.765, Lag order = 17, p-value = 0.01
alternative hypothesis: stationary

    \end{verbatim}

    
    \begin{Verbatim}[commandchars=\\\{\}]
Warning message in kpss.test(na.omit(SamplePlatinumPricesDecomposeA)):
"p-value greater than printed p-value"
    \end{Verbatim}

    
    \begin{verbatim}

	KPSS Test for Level Stationarity

data:  na.omit(SamplePlatinumPricesDecomposeA)
KPSS Level = 0.20715, Truncation lag parameter = 15, p-value = 0.1

    \end{verbatim}

    
    \begin{Verbatim}[commandchars=\\\{\}]
Warning message in kpss.test(na.omit(SamplePlatinumPricesDecomposeM)):
"p-value greater than printed p-value"
    \end{Verbatim}

    
    \begin{verbatim}

	KPSS Test for Level Stationarity

data:  na.omit(SamplePlatinumPricesDecomposeM)
KPSS Level = 0.10767, Truncation lag parameter = 15, p-value = 0.1

    \end{verbatim}

    
    \begin{Verbatim}[commandchars=\\\{\}]
Warning message in kpss.test(na.omit(SamplePlatinumPricesDecomposeSTL)):
"p-value greater than printed p-value"
    \end{Verbatim}

    
    \begin{verbatim}

	KPSS Test for Level Stationarity

data:  na.omit(SamplePlatinumPricesDecomposeSTL)
KPSS Level = 0.035836, Truncation lag parameter = 16, p-value = 0.1

    \end{verbatim}

    
    ADF teste göre p değeri çok küçük çıktığı için göre serinin durağan
olduğu söylenebilir. KPSS testine göre ise STL ile yapılan ayrıştırmanın
durağan olmadığını göstermektedir. Fakat toplamsal ve çarpımsal modele
göre yapılan ayrıştırmanın KPSS testinde p değeri büyük olduğu için
sıfır hipotezi reddedilemez ve durağan oldukları sonucuna varılır.

\subsection{Box-Jenkins Modelleri}\label{box-jenkins-modelleri}

Gecikmeli doğrusal ilişkiler yoluyla ortaya çıkabilen korelasyonun
getirilmesi, otoregresif (AR), hareketli ortalamalar(MA) otoregresif
hareketli ortalama (ARMA) modellerin önerilmesine yol açmaktadır. Ayrıca
durağan olmayan modeller için Box ve Jenkins (1970) tarafından yapılan
otoregresif entegre hareketli ortalama (ARIMA) modeli ortaya çıkmıştır.

\subsubsection{Durağan Zaman Serisi
Modelleri}\label{duraux11fan-zaman-serisi-modelleri}

\subsubsection{Otoregresif Modeller (AR)}\label{otoregresif-modeller-ar}

Bir zaman serisi modelinin en doğal formülasyonu, geçmiş değerleriyle
doğrusal bir ilişkiye sahip olması yani serinin kendisinin herhangi bir
gerilemesi ile arasında regresyon olmasıdır. Bu, otoregresif model ile
yapılır ve zaman serilerinin açıklamanın en kullanılan
modelidir.Otoregresif modeller, \(x_t\) serisinin mevcut değerinin,
geçmiş değerlerin, \(x_{t-1}\),\(x_{t-2}\),...,\(x_{t-p}\) nin bir
fonksiyonu olarak açıklanabileceği fikrine dayanmaktadır. Burada p,
mevcut değeri tahmin etmek için gerekli olan geçmiş adım sayısını
belirler. \(AR(p)\) aşağıdaki denklemlere göre geçmiş gözlemlerin
doğrusal bir kombinasyonuna dayanır: \[X_t=\sum_i a_iX_{t-i}+w_t\]
Buradaki \(a_i\) değerleri katsayılardır. \(w_t\) terimi bir Beyaz
Gürültü işleminden gelir. \(AR(p)\) modelleri sadece sabit zaman
serilerine uygulanabilir. Herhangi bir eğilim ve / veya mevsimsel
etkilerin öncelikle kaldırılması gerekir. Verilere bir \(AR(p)\)
modelinin p değerinin saptanması kısmi otokorelasyon fonksiyonunun(PACF)
analizine dayanır. PACF analizine göre ilk olarak makul görünen en basit
modeli denenir. PACF'nin 'cut-off' çizgisini kestiği tüm değerler olası
p değerleridir. İlk olarak en küçük p değerinden başlanarak diğer
değerler denenir.

\subsubsection{Kısmi Otokorelasyon
Fonksiyonu(PACF)}\label{kux131smi-otokorelasyon-fonksiyonupacf}

Genel olarak, kısmi bir korelasyon bir koşullu korelasyondur.Diğer
değişkenler grubunun değerlerini bildiğimiz ve hesaba kattığımız
varsayımı altında iki değişken arasındaki ilişkidir.\(Y\) ve \(x_3\)
arasındaki kısmi korelasyon, hem \(Y\) hem de \(x_3\)'ün \(x_1\) ve
\(x_2\) ile ilgili olduğunu dikkate alarak belirlenen değişkenler
arasındaki korelasyondur. N. dereceden Kısmi korelasyonu aşağıdaki
notasyon ile tanımlayabiliriz. {[}3{]}

\[\frac{Kovaryans(x_t,x_{t-n}|x_{t-1},x_{t-2},...x_{t-n+1})}{\sqrt{(Varyans(x_t|x_{t-1},x_{t-2},...x_{t-n+1})Varyans(x_{t-n}|x_{t-1},x_{t-2},...x_{t-n+1})}}\]

Aşağıda trend ve mevsimselliği ayrıştırıldıktan sonra kalan \(R_t\)
fonksiyonlarının PACF korelogramları hesaplanmıştır.

    \begin{Verbatim}[commandchars=\\\{\}]
{\color{incolor}In [{\color{incolor}8}]:} pacf\PY{p}{(}na.omit\PY{p}{(}PlatinumPricesDecomposeA\PY{o}{\PYZdl{}}random\PY{p}{)}\PY{p}{,}
             main\PY{o}{=}\PY{l+s}{\PYZsq{}}\PY{l+s}{Toplamsal Model için PACF\PYZsq{}}\PY{p}{)}
        pacf\PY{p}{(}na.omit\PY{p}{(}PlatinumPricesDecomposeM\PY{o}{\PYZdl{}}random\PY{p}{)}\PY{p}{,}
             main\PY{o}{=}\PY{l+s}{\PYZsq{}}\PY{l+s}{Çarpımsal Model için PACF\PYZsq{}}\PY{p}{)}
        pacf\PY{p}{(}na.omit\PY{p}{(}PlatinumPricesSTL\PY{o}{\PYZdl{}}time.series\PY{p}{[}\PY{p}{,}\PY{l+m}{3}\PY{p}{]}\PY{p}{)}\PY{p}{,}
             main\PY{o}{=}\PY{l+s}{\PYZsq{}}\PY{l+s}{STL Model için PACF\PYZsq{}}\PY{p}{)}
\end{Verbatim}


    \begin{center}
    \adjustimage{max size={0.9\linewidth}{0.9\paperheight}}{output_14_0.png}
    \end{center}
    { \hspace*{\fill} \\}
    
    \begin{center}
    \adjustimage{max size={0.9\linewidth}{0.9\paperheight}}{output_14_1.png}
    \end{center}
    { \hspace*{\fill} \\}
    
    \begin{center}
    \adjustimage{max size={0.9\linewidth}{0.9\paperheight}}{output_14_2.png}
    \end{center}
    { \hspace*{\fill} \\}
    
    Hesaplanan PACF korelogramlarına göre AR(p) modelinin p değerleri
toplamsal modele göre \{1,2,4,7\}'dir. Çarpımsal modele göre ise
\{1,2,3,4\}, stl modeline göre yapılan ayrıştırmaya göre ise
\{1,2,4,7\}'dir.

\subsubsection{Hareketli Ortalamalar
Modeli(MA)}\label{hareketli-ortalamalar-modelima}

Hareketli ortalama (MA) modeli, mevcut beyaz gürültü teriminin ve yakın
geçmişte geçen beyaz gürültü terimlerinin doğrusal bir kombinasyonu
şeklinde tanımlanabilir. Birçok açıdan hareketli ortalama modellerin
otoregresif modellere tamamlayıcısı olmaktadır. Yukarıda bahsedildiği
üzere, MA(q) modelinin \(E_t\) yani hata terimlerinin bir kombinasyonu
olduğu görülmektedir. Aşağıdaki notasyon ile ifade edilebilir;
\[x_t = E_t + θ_1E_{t−1} + θ_2E_{t−2} + · · · + θ_qE_{t-q}\] Buradaki
\(E_t\) değeri sıfır ortalama ve varyanslı beyaz gürültüdür. Verilere
bir \(MA(q)\) modelinin q değerinin saptanması otokorelasyon
fonksiyonunun(ACF) analizine dayanır.

\subsubsection{Otokorelasyon
Fonksiyonu(ACF)}\label{otokorelasyon-fonksiyonuacf}

Bir zaman serisi için otokorelasyon fonksiyonu (ACF), dizinin \(x_t\) ve
1, 2, 3 ve benzeri gecikmeler için dizinin gecikmeli değerleri
arasındaki korelasyonları verir. Gecikmiş değerler \(x_{t-1}\),
\(x_{t-2}\), \(x_{t-3}\) ve benzeri şekilde yazılabilir. ACF,\(x_t\)
ve\(x_{t-1}\), \(x_{t-2}\), \(x_{t-3}\),.. arasındaki korelasyonları
verir. \(X_t\) ile \(X_{t+k}\) arasındaki otokorelasyon;
\[Cov(X_{t+k},X_t)= \frac{Kovaryans(x_t,x_{t+k})}{\sqrt{(Varyans(x_t)Varyans(x_{t+k})}}\]

{[}3{]} Aşağıda trend ve mevsimselliği ayrıştırıldıktan sonra kalan
\(R_t\) fonksiyonlarının ACF korelogramları hesaplanmıştır.

    \begin{Verbatim}[commandchars=\\\{\}]
{\color{incolor}In [{\color{incolor}9}]:} acf\PY{p}{(}na.omit\PY{p}{(}PlatinumPricesDecomposeA\PY{o}{\PYZdl{}}random\PY{p}{)}\PY{p}{,}lag.max \PY{o}{=} \PY{l+m}{50000}\PY{p}{,}
            main\PY{o}{=}\PY{l+s}{\PYZsq{}}\PY{l+s}{Toplamsal Model için ACF\PYZsq{}}\PY{p}{)}
        acf\PY{p}{(}na.omit\PY{p}{(}PlatinumPricesDecomposeM\PY{o}{\PYZdl{}}random\PY{p}{)}\PY{p}{,}lag.max \PY{o}{=} \PY{l+m}{50000}\PY{p}{,}
            main\PY{o}{=}\PY{l+s}{\PYZsq{}}\PY{l+s}{Çarpımsal Model için ACF\PYZsq{}}\PY{p}{)}
        acf\PY{p}{(}na.omit\PY{p}{(}PlatinumPricesSTL\PY{o}{\PYZdl{}}time.series\PY{p}{[}\PY{p}{,}\PY{l+m}{3}\PY{p}{]}\PY{p}{)}\PY{p}{,}lag.max \PY{o}{=} \PY{l+m}{50000}\PY{p}{,}
            main\PY{o}{=}\PY{l+s}{\PYZsq{}}\PY{l+s}{STL Model için ACF\PYZsq{}}\PY{p}{)}
\end{Verbatim}


    \begin{center}
    \adjustimage{max size={0.9\linewidth}{0.9\paperheight}}{output_16_0.png}
    \end{center}
    { \hspace*{\fill} \\}
    
    \begin{center}
    \adjustimage{max size={0.9\linewidth}{0.9\paperheight}}{output_16_1.png}
    \end{center}
    { \hspace*{\fill} \\}
    
    \begin{center}
    \adjustimage{max size={0.9\linewidth}{0.9\paperheight}}{output_16_2.png}
    \end{center}
    { \hspace*{\fill} \\}
    
    ACF korelagramına bakılarak MA(q) modeli için q parametresi tahminlemesi
yapılamamaktadır. 

\subsubsection{ARMA Modelleri}\label{arima-modelleri} 

ARMA modelleri AR(p) ve MA(q) modellerinin
birlikte kullanılması durumundan ortaya çıkan model ARMA(p,q) olarak
adlandırılır. Önemi, çok daha geniş bir bağımlılık yapıları yelpazesinin
modellenmesinin mümkün olduğu ve bunların da karmaşık olduğu gerçeğinde
yatmaktadır. Çoğu zaman, bir ARMA(p,q) sadece AR veya MA süreçleriyle
bulunan p,q değerlerinden daha az sayıda parametre gerektirmektedir.
Burada model hem \{\(E_t\)\} kalan terimlerinin hem de \{\(X_t\)\}
önceki verilerin kombinasyonu şeklindedir. Bir ARMA (p,q) modeli
aşağıdaki notasyon ile ifade edilebilir:
\[x_t = a_1x_{t-1}+ a_2x_{t-2}+...+a_px_{t-p}+E_t + θ_1E_{t−1} + θ_2E_{t−2} + ... + θ_qE_{t-q}\]

\subsubsection{ARIMA Modelleri}\label{arima-modelleri}

Geniş bir dizi durağan olmayan seriyi içeren bu sınıfın bir genellemesi,
ARIMA süreçleri, yani birçok kez farklılaştıklarında ARMA işlemlerine
indirgeyen süreçler tarafından sağlanır. \{\(x_t\)\} serisi bir
ARMA(p,q) modelinden elde edilen bir seri olmak üzere, eğer \(d^{th}\)
dereceden farkı alınmasıyla elde edilen süreç ARIMA olarak adlandırılır.
\[Y_t=(1-B)^dX_t\] şeklinde ifade edilir. Buradaki B, 'backshift'
operatörüdür. R'da auto.arima fonksiyonu AIC, AICc veya BIC değerine
göre en iyi ARIMA modelinin oluşturulmasını sağlar.{[}3{]}

\subsubsection{Model kriterlerinin kullanılarak uygun ARIMA modelinin
seçimi}\label{model-kriterlerinin-kullanux131larak-uygun-arima-modelinin-seuxe7imi}

Model seçim kriterlerine örnek olarak AIC ve BIC verilebilir.

\paragraph{Akaike Bilgi Kriteri (Akaike Information Criteria,
AIC)}\label{akaike-bilgi-kriteri-akaike-information-criteria-aic}

Modelin kalanlarından kareler toplamı üzerinde örneklem büyüklüğü ve
değişken sayısını dikkate alır ve bir düzenleme yaparak elde edilen
değer sayesinde farklı modeller arasında en uygununu seçmeye yarayan bir
kriterdir. Akaike tarafından 1974 yılında kazandırılmış olan AIC değeri
her model için tahmin edilir ve bu değerin daha küçük olduğu modelin
daha uygun olduğu ifade edilir.AIC aşağıdaki formülle hesaplanabilir.

\[ AIC=-2 log(L)+ 2k \]

Formülde yer alan k sabit terim dahil parametre sayısı, n gözlem
sayısını ve L model için olabilirlik fonksiyonu maksimize değerini
göstermektedir.

\paragraph{Bayes bilgisi kriteri (Bayesian Information Criterion,
BIC)}\label{bayes-bilgisi-kriteri-bayesian-information-criterion-bic}

Doğrusal regresyonda seçilmiş model problemleri için BIC model seçim
kriterini kullanılır. AIC gibi farklı modeller arasında en uygununu
seçmeye yarayan bir kriterdir. BIC değeri küçük olan model daha
uygundur. AIC aşağıdaki formülle hesaplanabilir.

\[ BIC=-2 log(L)+ k log(n) \]

Model seçim kriterleri olarak minimum AIC ve BIC değerleri kullanılır.
En uygun model, minimum AIC ve BIC'ye göre seçilir. Aşağıda toplamsal,
çarpımsal ve stl modellerine göre yapılan ayrıştırmanın kalanı ile
herhangi bit işlem yapılmadan saf veriye auto.arima uygulanmıştır.

    \begin{Verbatim}[commandchars=\\\{\}]
{\color{incolor}In [{\color{incolor}11}]:} auto.arima\PY{p}{(}na.omit\PY{p}{(}PlatinumPricesDecomposeM\PY{o}{\PYZdl{}}random\PY{p}{)}\PY{p}{)}
         auto.arima\PY{p}{(}na.omit\PY{p}{(}PlatinumPricesDecomposeA\PY{o}{\PYZdl{}}random\PY{p}{)}\PY{p}{)}
         auto.arima\PY{p}{(}na.omit\PY{p}{(}PlatinumPricesSTL\PY{o}{\PYZdl{}}time.series\PY{p}{[}\PY{p}{,}\PY{l+m}{3}\PY{p}{]}\PY{p}{)}\PY{p}{)}
         auto.arima\PY{p}{(}na.omit\PY{p}{(}PlatinumPrices\PY{p}{[}\PY{p}{,}\PY{l+m}{2}\PY{p}{]}\PY{p}{)}\PY{p}{)}
\end{Verbatim}


    
    \begin{verbatim}
Series: na.omit(PlatinumPricesDecomposeM$random) 
ARIMA(1,0,3) with non-zero mean 

Coefficients:
         ar1     ma1      ma2      ma3    mean
      0.9856  0.0591  -0.0209  -0.0290  0.9883
s.e.  0.0021  0.0124   0.0123   0.0118  0.0169

sigma^2 estimated as 0.0004002:  log likelihood=16807.26
AIC=-33602.51   AICc=-33602.5   BIC=-33561.61
    \end{verbatim}

    
    
    \begin{verbatim}
Series: na.omit(PlatinumPricesDecomposeA$random) 
ARIMA(2,0,2) with zero mean 

Coefficients:
         ar1     ar2     ma1     ma2
      0.0955  0.8746  0.9468  0.0713
s.e.  0.0628  0.0620  0.0635  0.0125

sigma^2 estimated as 75.09:  log likelihood=-24124.78
AIC=48259.57   AICc=48259.57   BIC=48293.65
    \end{verbatim}

    
    
    \begin{verbatim}
Series: na.omit(PlatinumPricesSTL$time.series[, 3]) 
ARIMA(2,0,2) with zero mean 

Coefficients:
         ar1     ar2     ma1     ma2
      0.0997  0.8674  0.9383  0.0711
s.e.  0.0585  0.0576  0.0593  0.0122

sigma^2 estimated as 76.8:  log likelihood=-25507.05
AIC=51024.11   AICc=51024.12   BIC=51058.45
    \end{verbatim}

    
    
    \begin{verbatim}
Series: na.omit(PlatinumPrices[, 2]) 
ARIMA(0,1,3) 

Coefficients:
         ma1     ma2      ma3
      0.0458  0.0121  -0.0462
s.e.  0.0118  0.0117   0.0114

sigma^2 estimated as 81.76:  log likelihood=-25724.6
AIC=51457.2   AICc=51457.2   BIC=51484.67
    \end{verbatim}

    
    Veri içerisindeki trend, mevsimsel etki ve üssel dağılım konseptleri
çıkarıldıktan sonra elde kalan verinin beyaz gürültü olması
beklenmektedir. Decompose işlemi ile hem trend hem de mevsimsellik etki
veriden uzaklaştırılmıştır. Arima modelleri ise veriden üssel etkilerin
çıkarılmasını sağlar.

Uyguladığımız arima modellerinden AIC,BIC değeri mutlak değerce en düşük
olan multiplicative veriye uygulanan p=1, q=0,r=3 modeli seçilmiştir.
Veri seti 1 gün önceye bağlı olarak ve 3 günün ortalamasını alarak
ilerlemektedir. Herhangi bir fark alınmadığı için durağan bir veri
olduğu söylenebilir. Verideki durağanlık daha önceden yapılmış olan
ayrıştırma işlemleriyle uzaklaştırılmıştır.

ARIMA modeli kullanılarak üstel etkilerin çıkarılması sonucu kalan
seriye aşağıdaki şekilde ulaşılabilir. Ayrıca bu kalan değerin beyaz
gürültü fonksiyonu olması beklenmektedir. Eğer kalan veri beyaz gürültü
ise, veriye doğru model uygulanarak, gerekli anomalilerin çıkarıldığı
sonucuna varılmaktadır.

    \begin{Verbatim}[commandchars=\\\{\}]
{\color{incolor}In [{\color{incolor}12}]:} PlatinumPricesArima\PY{o}{\PYZlt{}\PYZhy{}}arima\PY{p}{(}na.omit\PY{p}{(}PlatinumPricesDecomposeM\PY{o}{\PYZdl{}}random\PY{p}{)}\PY{p}{,}
                                    order\PY{o}{=}\PY{k+kt}{c}\PY{p}{(}\PY{l+m}{1}\PY{p}{,}\PY{l+m}{0}\PY{p}{,}\PY{l+m}{3}\PY{p}{)}\PY{p}{)}
         WhiteNoise\PY{o}{\PYZlt{}\PYZhy{}}PlatinumPricesArima\PY{o}{\PYZdl{}}residuals
         plot\PY{p}{(}WhiteNoise\PY{p}{)}
         \PY{k+kp}{summary}\PY{p}{(}PlatinumPricesArima\PY{p}{)}
\end{Verbatim}


    \begin{Verbatim}[commandchars=\\\{\}]

Call:
arima(x = na.omit(PlatinumPricesDecomposeM\$random), order = c(1, 0, 3))

Coefficients:
         ar1     ma1      ma2      ma3  intercept
      0.9856  0.0591  -0.0209  -0.0290     0.9883
s.e.  0.0021  0.0124   0.0123   0.0118     0.0169

sigma\^{}2 estimated as 0.0003999:  log likelihood = 16807.26,  aic = -33602.51

Training set error measures:
                        ME       RMSE        MAE         MPE    MAPE      MASE
Training set -2.650106e-05 0.01999801 0.01345905 -0.04244688 1.35439 0.9949594
                     ACF1
Training set 2.833433e-05

    \end{Verbatim}

    \begin{center}
    \adjustimage{max size={0.9\linewidth}{0.9\paperheight}}{output_20_1.png}
    \end{center}
    { \hspace*{\fill} \\}
    
    Kalan verinin grafiğindeki bantın genişliği riski ifade etmektedir. Risk
neredeyse tüm zamanlarda aynıdır. Yukarıdaki arima modelin özet çıktısı
doğru modelin uygulandığını hata ölçüm parametrelerinden göstermektedir.
RMSE, MAE,MPE, MAPE gibi ölçüm değerleri uygulanan model için düşük
çıkmıştır. Ayrıca özet fonksiyonunu çıktısındaki ACF değeri de modelin
doğru olduğunu göstermektedir. Tüm bunlar da uygulanan arima(1,0,3)
modelinin aslında doğru model olduğunu ve modelden çıkarılan değerlerin
doğru sonuçlar verdiğini göstermektedir. Aslında arima(1,0,3) modeli
herhangi bir fark alınmadığı için arma(1,3) modelinin benzeridir.

Arima modeli uygulandıktan sonra kalanından elde edilmesi beklenen beyaz
gürültü fonksiyonu olup olmadığı, normallik testleriyle aşağıdaki R kodu
ile yapılabilir;

    \begin{Verbatim}[commandchars=\\\{\}]
{\color{incolor}In [{\color{incolor}13}]:} SampleWN\PY{o}{\PYZlt{}\PYZhy{}}\PY{k+kp}{sample}\PY{p}{(}WhiteNoise\PY{p}{,}\PY{l+m}{5000}\PY{p}{)}
         shapiro.test\PY{p}{(}SampleWN\PY{p}{)}
         ks.test\PY{p}{(}WhiteNoise\PY{p}{,} \PY{l+s}{\PYZsq{}}\PY{l+s}{pnorm\PYZsq{}}\PY{p}{,}mean\PY{o}{=}\PY{l+m}{0}\PY{p}{,}sd\PY{o}{=}\PY{k+kp}{sqrt}\PY{p}{(}var\PY{p}{(}WhiteNoise\PY{p}{)}\PY{p}{)}\PY{p}{)}
\end{Verbatim}


    
    \begin{verbatim}

	Shapiro-Wilk normality test

data:  SampleWN
W = 0.90633, p-value < 2.2e-16

    \end{verbatim}

    
    
    \begin{verbatim}

	One-sample Kolmogorov-Smirnov test

data:  WhiteNoise
D = 0.080609, p-value < 2.2e-16
alternative hypothesis: two-sided

    \end{verbatim}

    
    Çarpımsal modelden ARIMA(1,0,3) çıkarılmadan önce yapılan normallik
testlerinin sonuçları aşağıda verilmiştir.

    \begin{Verbatim}[commandchars=\\\{\}]
{\color{incolor}In [{\color{incolor}14}]:} shapiro.test\PY{p}{(}SamplePlatinumPricesDecomposeM\PY{p}{)}
         ks.test\PY{p}{(}PlatinumPricesDecomposeM\PY{o}{\PYZdl{}}random\PY{p}{,}
                 \PY{l+s}{\PYZsq{}}\PY{l+s}{pnorm\PYZsq{}}\PY{p}{,}
                 mean\PY{o}{=}\PY{l+m}{0}\PY{p}{,}
                 sd\PY{o}{=}\PY{k+kp}{sqrt}\PY{p}{(}var\PY{p}{(}na.omit\PY{p}{(}PlatinumPricesDecomposeM\PY{o}{\PYZdl{}}random\PY{p}{)}\PY{p}{)}\PY{p}{)}\PY{p}{)}
\end{Verbatim}


    
    \begin{verbatim}

	Shapiro-Wilk normality test

data:  SamplePlatinumPricesDecomposeM
W = 0.96362, p-value < 2.2e-16

    \end{verbatim}

    
    
    \begin{verbatim}

	One-sample Kolmogorov-Smirnov test

data:  PlatinumPricesDecomposeM$random
D = 1, p-value < 2.2e-16
alternative hypothesis: two-sided

    \end{verbatim}

    
    Shapiro-Wilk normallik testine göre W değeri küçülmüştür. Bu da
arima(1,0,3) modeli uygulandıktan sonra kalan verinin normallikten
uzaklaştığını göstermektedir.Fakat modelin hala normal olduğu W test
istatistiğinden görülmektedir.Kolmogorov-Smirnov testine göre zaman
serisinin kalanının normal dağılıma daha çok yaklaştığı D test
istatistiğinden görülmektedir. D değeri küçüldüğünden seri normalliğe
daha da yaklaşmıştır. Kolmogorov testi tüm veri setini test ettiği için
Shapiro-Wilk testine göre daha doğru sonuçlar vermektedir.

AIC ve BIC değerleri aynı ayrıştırma modeli ile ayrıştırılmış (örneğin
ikisi de toplamsal model olarak) fakat farklı arima modelleri uygulanmış
veriler üzerinde daha doğru bir seçim kriteri oluşturmaktadır. Bu yüzden
yukarıdaki tüm arima modelleri için auto.arima fonksiyonun sonuçları
değerlendirilmeli ve normallik testleri uygulanmalıdır. Eğer bu
çıktılardan herhangi biri çarpımsal modele uyguladığımız ARIMA(1,0,3)
modelininden elde edilen sonuçlarından daha iyi sonuçlara sahip ise o
modelin daha uygun olduğu söylenebilir. Daha sonra hangisinin beyaz
gürültüye daha yakın olduğu saptanmalıdır. Aşağıda bu işlemler
gösterilmiştir.

Toplamsal Model;

    \begin{Verbatim}[commandchars=\\\{\}]
{\color{incolor}In [{\color{incolor}16}]:} PlatinumPricesArimaToplamsal\PY{o}{\PYZlt{}\PYZhy{}}arima\PY{p}{(}na.omit\PY{p}{(}PlatinumPricesDecomposeA\PY{o}{\PYZdl{}}random\PY{p}{)}\PY{p}{,}
                                             order\PY{o}{=}\PY{k+kt}{c}\PY{p}{(}\PY{l+m}{2}\PY{p}{,}\PY{l+m}{0}\PY{p}{,}\PY{l+m}{2}\PY{p}{)}\PY{p}{)}
         \PY{k+kp}{summary}\PY{p}{(}PlatinumPricesArimaToplamsal\PY{p}{)}
         WhiteNoise2\PY{o}{\PYZlt{}\PYZhy{}}PlatinumPricesArimaToplamsal\PY{o}{\PYZdl{}}residuals
         SampleWN2\PY{o}{\PYZlt{}\PYZhy{}}\PY{k+kp}{sample}\PY{p}{(}WhiteNoise2\PY{p}{,}\PY{l+m}{5000}\PY{p}{,}replace \PY{o}{=} \PY{k+kc}{TRUE}\PY{p}{)}
         shapiro.test\PY{p}{(}SampleWN2\PY{p}{)}
         ks.test\PY{p}{(}WhiteNoise2\PY{p}{,} \PY{l+s}{\PYZsq{}}\PY{l+s}{pnorm\PYZsq{}}\PY{p}{,}mean\PY{o}{=}\PY{l+m}{0}\PY{p}{,}sd\PY{o}{=}\PY{k+kp}{sqrt}\PY{p}{(}var\PY{p}{(}WhiteNoise2\PY{p}{)}\PY{p}{)}\PY{p}{)}
\end{Verbatim}


    \begin{Verbatim}[commandchars=\\\{\}]

Call:
arima(x = na.omit(PlatinumPricesDecomposeA\$random), order = c(2, 0, 2))

Coefficients:
         ar1     ar2     ma1     ma2  intercept
      0.0006  0.9681  1.0407  0.0634    -1.2111
s.e.  0.0150  0.0148  0.0191  0.0124     7.0326

sigma\^{}2 estimated as 75.07:  log likelihood = -24125.62,  aic = 48263.23

Training set error measures:
                     ME     RMSE      MAE     MPE     MAPE      MASE
Training set -0.0100795 8.664197 5.563736 43.3694 103.5062 0.9942941
                     ACF1
Training set 0.0004094616

    \end{Verbatim}

    
    \begin{verbatim}

	Shapiro-Wilk normality test

data:  SampleWN2
W = 0.89915, p-value < 2.2e-16

    \end{verbatim}

    
    
    \begin{verbatim}

	One-sample Kolmogorov-Smirnov test

data:  WhiteNoise2
D = 0.1105, p-value < 2.2e-16
alternative hypothesis: two-sided

    \end{verbatim}

    
    STL modeli;

    \begin{Verbatim}[commandchars=\\\{\}]
{\color{incolor}In [{\color{incolor}17}]:} PlatinumPricesArimaSTL\PY{o}{\PYZlt{}\PYZhy{}}arima\PY{p}{(}na.omit\PY{p}{(}PlatinumPricesSTL\PY{o}{\PYZdl{}}time.series\PY{p}{[}\PY{p}{,}\PY{l+m}{3}\PY{p}{]}\PY{p}{)}\PY{p}{,}
                                       order\PY{o}{=}\PY{k+kt}{c}\PY{p}{(}\PY{l+m}{2}\PY{p}{,}\PY{l+m}{0}\PY{p}{,}\PY{l+m}{2}\PY{p}{)}\PY{p}{)}
         \PY{k+kp}{summary}\PY{p}{(}PlatinumPricesArimaSTL\PY{p}{)}
         WhiteNoise3\PY{o}{\PYZlt{}\PYZhy{}}PlatinumPricesArimaSTL\PY{o}{\PYZdl{}}residuals
         SampleWN3\PY{o}{\PYZlt{}\PYZhy{}}\PY{k+kp}{sample}\PY{p}{(}WhiteNoise3\PY{p}{,}\PY{l+m}{5000}\PY{p}{)}
         shapiro.test\PY{p}{(}SampleWN3\PY{p}{)}
         ks.test\PY{p}{(}WhiteNoise3\PY{p}{,} \PY{l+s}{\PYZsq{}}\PY{l+s}{pnorm\PYZsq{}}\PY{p}{,}mean\PY{o}{=}\PY{l+m}{0}\PY{p}{,}sd\PY{o}{=}\PY{k+kp}{sqrt}\PY{p}{(}var\PY{p}{(}WhiteNoise3\PY{p}{)}\PY{p}{)}\PY{p}{)}
\end{Verbatim}


    \begin{Verbatim}[commandchars=\\\{\}]

Call:
arima(x = na.omit(PlatinumPricesSTL\$time.series[, 3]), order = c(2, 0, 2))

Coefficients:
         ar1     ar2     ma1     ma2  intercept
      0.0997  0.8674  0.9382  0.0711    -0.8504
s.e.  0.0583  0.0575  0.0591  0.0122     6.2980

sigma\^{}2 estimated as 76.76:  log likelihood = -25507.03,  aic = 51026.07

Training set error measures:
                     ME   RMSE      MAE      MPE     MAPE      MASE
Training set 0.01563383 8.7611 5.587638 -11.8678 87.51329 0.9948034
                      ACF1
Training set -0.0009323478

    \end{Verbatim}

    
    \begin{verbatim}

	Shapiro-Wilk normality test

data:  SampleWN3
W = 0.8832, p-value < 2.2e-16

    \end{verbatim}

    
    
    \begin{verbatim}

	One-sample Kolmogorov-Smirnov test

data:  WhiteNoise3
D = 0.11626, p-value < 2.2e-16
alternative hypothesis: two-sided

    \end{verbatim}
\pagebreak
    
    Saf veri seti;

    \begin{Verbatim}[commandchars=\\\{\}]
{\color{incolor}In [{\color{incolor}18}]:} PlatinumPricesArimaPure\PY{o}{\PYZlt{}\PYZhy{}}arima\PY{p}{(}na.omit\PY{p}{(}PlatinumPrices\PY{p}{[}\PY{p}{,}\PY{l+m}{2}\PY{p}{]}\PY{p}{)}\PY{p}{,}
                                        order\PY{o}{=}\PY{k+kt}{c}\PY{p}{(}\PY{l+m}{0}\PY{p}{,}\PY{l+m}{1}\PY{p}{,}\PY{l+m}{3}\PY{p}{)}\PY{p}{)}
         \PY{k+kp}{summary}\PY{p}{(}PlatinumPricesArimaPure\PY{p}{)}
         WhiteNoise4\PY{o}{\PYZlt{}\PYZhy{}}PlatinumPricesArimaPure\PY{o}{\PYZdl{}}residuals
         SampleWN4\PY{o}{\PYZlt{}\PYZhy{}}\PY{k+kp}{sample}\PY{p}{(}WhiteNoise4\PY{p}{,}\PY{l+m}{5000}\PY{p}{)}
         shapiro.test\PY{p}{(}SampleWN4\PY{p}{)}
         ks.test\PY{p}{(}WhiteNoise4\PY{p}{,} \PY{l+s}{\PYZsq{}}\PY{l+s}{pnorm\PYZsq{}}\PY{p}{,}mean\PY{o}{=}\PY{l+m}{0}\PY{p}{,}sd\PY{o}{=}\PY{k+kp}{sqrt}\PY{p}{(}var\PY{p}{(}WhiteNoise4\PY{p}{)}\PY{p}{)}\PY{p}{)}
\end{Verbatim}


    \begin{Verbatim}[commandchars=\\\{\}]

Call:
arima(x = na.omit(PlatinumPrices[, 2]), order = c(0, 1, 3))

Coefficients:
         ma1     ma2      ma3
      0.0458  0.0121  -0.0462
s.e.  0.0118  0.0117   0.0114

sigma\^{}2 estimated as 81.73:  log likelihood = -25724.6,  aic = 51457.2

Training set error measures:
                     ME     RMSE      MAE         MPE     MAPE      MASE
Training set -0.1186566 9.039685 5.439082 -0.04801586 1.339236 0.9996372
                      ACF1
Training set -0.0007836219

    \end{Verbatim}

    
    \begin{verbatim}

	Shapiro-Wilk normality test

data:  SampleWN4
W = 0.86965, p-value < 2.2e-16

    \end{verbatim}

    
    
    \begin{verbatim}

	One-sample Kolmogorov-Smirnov test

data:  WhiteNoise4
D = 0.13969, p-value < 2.2e-16
alternative hypothesis: two-sided

    \end{verbatim}

    
    Ayrıştırmalara uygulanan normallik testlerine göre bu çıktılardan
herhangi biri çarpımsal modele uyguladığımız ARIMA(1,0,3) modelininden
elde edilen sonuçlarından daha iyi sonuca sahip değildir. Ayrıca Summary
fonksiyonunun çıktısındaki hata ölçüm değerleri ile AIC değeri
modellerin çarpımsal modele uygulanan Arima(1,0,3) modelinin daha doğru
sonuçlar verdiğini göstermektedir.

\paragraph{Sonuç}\label{sonuuxe7}

Yukarıda yapılan testlere göre en iyi sonucun AIC ve BIC değerine göre
en iyi sonucu veren çarpımsal modele uygulanan arima(1,0,3) modeliyle
yapılan ayrıştırmaya göre olduğu görülmektedir. Veri setindeki tüm
anomaliler çıkarıldıktan sonra kalan verinin beyaz gürültü olması
beklenir. Beyaz gürültü fonksiyonu normal dağılıma sahiptir. Kalan
verinin gürültü datası olup olmadığı test edilmiş ve gürültü datası
olduğu sonucuna varılmıştır. Bu da veriye uygulanan ayrıştırma
modellerinin doğru bir şekilde uygulandığını göstermektedir.

    \section{Tahminleme}\label{tahminleme}

Zaman serileri analizi ile temel hedeflerden biri, verilerin gelecekteki
durumunu gösteren tahminler üretmektir.Bu tahminleme extrapolasyonla
yapılır. Ekstrapolasyon her zaman doğru sonuçlar vermeyebilir ve yanlış
sonuçlara yol açabilir. Tahminleme de aynı şekilde doğru
yapılamayabilir. Sinyal içeren seriler gürültüye göre daha güçlü olduğu
seriler oldukları için tahminlemenin doğru yapılması mümkündür. Bununla
birlikte, gürültülü seriler için, tahminlerde büyük bir belirsizlik
vardır ve bunlar çok kısa bir aralık için en güvenilirdir.
Yukarıdakilerden yola çıkarak, belirsizliklerin ana kaynağının sürecin
içindeki yenilikler olduğu söylenebilir. Doğru tahminleme yapabilmek
için veri üretme sürecinin zaman içinde değişmediğinden, yani geçmişte
gözlemlendiği gibi gelecekte devam edeceğinden emin olunmalıdır. Doğru
model uygulansa dahi parametreler arasındaki ek belirsizlikler doğru
tahminlemeyi etkilemektedir.

Zaman serisi tahminlemelerine ilk olarak, durağan süreçleri tahmin etmek
için kullanılan AR, MA ve ARMA süreçleriörnek olarak gösterilebilir.
Trend içeren durağan olmayan seriler için ise ARIMA modeli ile
tahminleme yapılabilir.

\subsection{ARIMA ile Tahminleme}\label{arima-ile-tahminleme}

Kalan \(R_t\) yani Beyaz Gürültü gibi göründüğü bir zaman dizisindeki
tahminleme ARIMA modeli ile yapılabilir. Bu koşullar altında, tahminler
kolayca hesaplanabilir

Uygun ARIMA modelinin tanımlanan parametrelerin başarılı bir şekilde
tahmin edildiği ve artıkların gerekli özellikleri sergilediği, yani
Beyaz Gürültü gibi göründüğü bir zaman dizisi verildiğini varsayıyoruz.
Bu koşullar altında, tahminler kolayca hesaplanabilir.

Bu, istediğimiz bir \(ARIMA (p,q,r)\) modelinden herhangi bir tahmin
üretmemizi sağlar. Bu prosedür aynı zamanda Box-Jenkins prosedürü olarak
da bilinmektedir. R'da kullanılan, arima ()'dan sonra uygulanan
forecast() fonksiyonunda Box-Jenkins şeması arka planda çalışmaktadır.

Verilerinin modele uyarlanması ve 30 adımlık tahminlerin üretilmesi için
R komutları aşağıdaki gibidir. Arima() orderleri belli modeli oluşturur
ve forecast() fonksiyonu belirlenen modele göre verilen parametre kadar
tahmin üretir. Oluşturulan tahminlemenin görselleştirilmesi için
aşağıdaki R kodundan yararlanılabilir.İlk aşamada yaptığımız durağan
olmayan zaman serileri için ayrıştırma aşamalarından yararlanılarak
trend, mevsimsellik içermeyen seriler için tahmin üretilebilecektir.
Yani çarpımsal ayrıştırmanın arima(1,0,3) modeline göre tahminlemesi
yapılacaktır. Son 30 veriye kadar model uygulanacak ve daha sonra 30
adet tahmin yapılıp gerçek değerlerle tahminler karşılaştırılacatır.
{[}4{]}

    \begin{Verbatim}[commandchars=\\\{\}]
{\color{incolor}In [{\color{incolor}19}]:} m\PY{o}{\PYZlt{}\PYZhy{}}\PY{k+kp}{length}\PY{p}{(}na.omit\PY{p}{(}PlatinumPricests\PY{p}{)}\PY{p}{)}
         PlatinumPricesArimaForecast\PY{o}{\PYZlt{}\PYZhy{}}arima\PY{p}{(}na.omit\PY{p}{(}PlatinumPricests\PY{p}{)}\PY{p}{[}\PY{l+m}{1}\PY{o}{:}\PY{p}{(}m\PY{l+m}{\PYZhy{}29}\PY{p}{)}\PY{p}{]}\PY{p}{,}
                                            order\PY{o}{=}\PY{k+kt}{c}\PY{p}{(}\PY{l+m}{1}\PY{p}{,}\PY{l+m}{0}\PY{p}{,}\PY{l+m}{3}\PY{p}{)}\PY{p}{)}
         tahmin \PY{o}{\PYZlt{}\PYZhy{}} forecast\PY{p}{(}PlatinumPricesArimaForecast\PY{p}{,}\PY{l+m}{30}\PY{p}{)}
         plot\PY{p}{(}tahmin\PY{p}{)}
\end{Verbatim}


    \begin{center}
    \adjustimage{max size={0.9\linewidth}{0.9\paperheight}}{output_33_0.png}
    \end{center}
    { \hspace*{\fill} \\}
    
    Aşağıda gerçek veriler ve arima modeliyle yapılan tahmin ile yapılan
veriler için ortalama hata hesaplanmıştır. Bu değer tahminlemelerin
karşılaştırılması için kullanılır.

    Keşif analizi yaparken bulduğumuz uygun arima modeli olan (1,0,3) modeli
uygulanarak yukarıdaki tahminleme yapılmıştır.

\subsection{Holt-Winters Yöntemi ile
Tahminleme}\label{holt-winters-yuxf6ntemi-ile-tahminleme}

Eğilim ve mevsimsellik sergileyen seriler için yapılan bir tahminleme
yöntemidir. Holt-Winters mevsimsel yöntemi, tahmin denklemini ve üç
yumuşatma denklemini içerir. Holt-Winters yöntemi toplamsal ve çarpımsal
olarak ikiye ayrılır. Toplamsal yöntemle, mevsimsel bileşen, gözlemlenen
serilerin ölçeğinde mutlak terimlerle ifade edilir ve seviye
denkleminde, mevsimsel bileşenin çıkarılmasıyla seri mevsimsel olarak
ayarlanır. Çarpımsal yöntemle, mevsimsel bileşen göreceli olarak ifade
edilir. Seri mevsimsel bileşen tarafından bölünerek mevsimsel olarak
ayarlanır.

\subsubsection{Toplamsal Holt-Winters
Yöntemi}\label{toplamsal-holt-winters-yuxf6ntemi}

Toplamsal model için kullanılan form aşağıdaki gibidir.
\[ a_t=\alpha(x_t-s{t-p})-(1-\alpha)(a_{t-1}+b_t{t-1})\]
\[ b_t=\beta(a_t-a_{t-1})-(1-\beta)(b_{t-1})\]
\[ s_t=\gamma(x_t-a_t)-(1-\gamma)(s{t-p})\]

Buradaki \(a_t\),seviye, \(b_t\) eğilim, \(s_t\) mevsimsel etkidir.
Seviye, eğim ve mevsimi hedefleyen üç düzeltme parametresi
\(\alpha\),\(\beta\),\(\gamma\)'dır.

İlk güncelleme denklemi, gözlemlediğimiz uygun mevsimsel etkinin mevcut
tahminiyle, son gözlemimizin ağırlıklı ortalamasını alır ve seviyenin
bir önceki adımdaki tahminini ifade eder.

İkinci güncelleme denklemi, mevcut seviye ve bir önceki seviye
arasındaki farkın t-1 zamanında tahmini eğim ile ağırlıklı ortalamasını
alır. Bu yalnızca mevcutsa hesaplanabilir.

Son olarak, mevsimsel terim için, aynı birim için mevsimsel terimin
önceki tahmini ile gözlem ve seviye arasındaki farkın ağırlıklı
ortalaması alınarak, t-p zamanında yapılan bir başka tahmin daha elde
edilir.

\subsubsection{Çarpımsal Holt-Winters
Yöntemi}\label{uxe7arpux131msal-holt-winters-yuxf6ntemi}

Çarpımsal model için kullanılan form aşağıdaki gibidir.
\[ a_n=\alpha \frac{x_n}{s_{n-p}}-(1-\alpha)(a_{n-1}+b_{n-1})\]
\[ b_n=\beta(a_n-a_{n-1})-(1-\beta)(b_{n-1})\]
\[ s_t=\gamma \frac{x_n}{a_n}-(1-\gamma)(s{n-p})\]

Buradaki ilk denklem seviye denklemi, ikinci denklem eğilim ve son
denklem mevsimsellik bileşenlerinin hesaplanmasında kullanılır.
\(\alpha\),\(\beta\),\(\gamma\), üç düzeltme parametresidir.

R fonksiyonu HoltWinters () uygulandığında, başlangıç değerleri
decompose () prosedüründen elde edilir ve uygun seviye, eğilim ve
mevsimsellik için mevcut tahminleri içerir. Daha önceden yaptığımız
analiz sonucuna göre veri setinin çarpımsal modelle daha iyi sonuçlar
verdiği gözlenmişti. Bu yüzden Holt-Winters tahminlemesinde çarpımsal
model kullanılmıştır.

Mevsimsellik ve trend içeren bir zaman serisinin tahmininde kullanılacak
Holt Winters parametreleri aşağıdaki şekilde elde edilebilir.{[}1{]}

    \begin{Verbatim}[commandchars=\\\{\}]
{\color{incolor}In [{\color{incolor}20}]:} PPHW \PY{o}{\PYZlt{}\PYZhy{}} HoltWinters\PY{p}{(}ts\PY{p}{(}na.omit\PY{p}{(}PlatinumPricests\PY{p}{)}\PY{p}{[}\PY{l+m}{1}\PY{o}{:}\PY{p}{(}m\PY{l+m}{\PYZhy{}29}\PY{p}{)}\PY{p}{]}\PY{p}{,}
                             freq\PY{o}{=}\PY{l+m}{365}\PY{p}{)}\PY{p}{,} 
                             seasonal \PY{o}{=} \PY{l+s}{\PYZdq{}}\PY{l+s}{mult\PYZdq{}}\PY{p}{)}
         PPHW\PY{o}{\PYZdl{}}alpha
         PPHW\PY{o}{\PYZdl{}}\PY{k+kp}{gamma}
         PPHW\PY{o}{\PYZdl{}}\PY{k+kp}{beta}
\end{Verbatim}


    \textbf{alpha:} 0.892454361002961

    
    \textbf{gamma:} 1

    
    \textbf{beta:} 0.00123075437713089

    
    Holt-Winters yönteminin alfa,gama ve beta sayıları yukarıda
gösterilmiştir. Katsayı değerleri (n zamanında), yukarıda verilen
formülle bu serilerden tahmin yapmak için kullanılanlardır. 30 basamaklı
bir tahmin tahmin üretmek için aşağıdaki R komutundan yararlanılabilir.

    \begin{Verbatim}[commandchars=\\\{\}]
{\color{incolor}In [{\color{incolor}21}]:} plot\PY{p}{(}PPHW\PY{p}{)}
         tahminHW\PY{o}{\PYZlt{}\PYZhy{}}predict\PY{p}{(}PPHW\PY{p}{,} n.ahead\PY{o}{=}\PY{l+m}{30}\PY{p}{)}
         lines\PY{p}{(}tahminHW\PY{p}{,} col\PY{o}{=}\PY{l+s}{\PYZdq{}}\PY{l+s}{blue\PYZdq{}}\PY{p}{,} lty\PY{o}{=}\PY{l+m}{4}\PY{p}{)}
\end{Verbatim}


    \begin{center}
    \adjustimage{max size={0.9\linewidth}{0.9\paperheight}}{output_38_0.png}
    \end{center}
    { \hspace*{\fill} \\}
    
    \subsubsection{Uygun Tahminleme Yönteminin
Seçilmesi}\label{uygun-tahminleme-yuxf6nteminin-seuxe7ilmesi}

Yukarıdaki ortalama hata karelerine göre arima ile yapılan tahminlemenin
çok daha yakın sonuçlar verdiği gözlenmiştir. Ayrıca accuracy()
fonksiyonu da tahminin doğru yapılıp yapılmadığını test etmektedir.
Bunun için öncelikle saf veri setinin son 30 girdisini window()
fonksiyonu ile oluşturmamız gerekmektedir. Daha sonra Accuracy()
fonksiyonu ile tahmin ve gerçek veriler karşılaştırılır.{[}5{]}

    \begin{Verbatim}[commandchars=\\\{\}]
{\color{incolor}In [{\color{incolor}22}]:} PP1Last30 \PY{o}{\PYZlt{}\PYZhy{}} window\PY{p}{(}na.omit\PY{p}{(}PlatinumPricests\PY{p}{)}\PY{p}{[}\PY{p}{(}m\PY{l+m}{\PYZhy{}29}\PY{p}{)}\PY{o}{:}m\PY{p}{]}\PY{p}{)}
         tahminHW \PY{o}{\PYZlt{}\PYZhy{}} window\PY{p}{(}tahminHW\PY{p}{)}
         \PY{k+kp}{print}\PY{p}{(}\PY{l+s}{\PYZdq{}}\PY{l+s}{Accuracy Arima\PYZdq{}}\PY{p}{)}
         accuracy\PY{p}{(}tahmin\PY{o}{\PYZdl{}}\PY{k+kp}{mean}\PY{p}{,} PP1Last30\PY{p}{)}
         \PY{k+kp}{print}\PY{p}{(}\PY{l+s}{\PYZdq{}}\PY{l+s}{Accuracy HW\PYZdq{}}\PY{p}{)}
         accuracy\PY{p}{(}tahminHW\PY{p}{,} PP1Last30\PY{p}{)}
\end{Verbatim}


    \begin{Verbatim}[commandchars=\\\{\}]
[1] "Accuracy Arima"

    \end{Verbatim}

    \begin{tabular}{r|lllllll}
  & ME & RMSE & MAE & MPE & MAPE & ACF1 & Theil's U\\
\hline
	Test set & 3.046062  & 4.748621  & 4.03789   & 2.348815  & 3.186527  & 0.9369171 & 4.110629 \\
\end{tabular}


    
    \begin{Verbatim}[commandchars=\\\{\}]
[1] "Accuracy HW"

    \end{Verbatim}

    \begin{tabular}{r|lllllll}
  & ME & RMSE & MAE & MPE & MAPE & ACF1 & Theil's U\\
\hline
	Test set & 6.746535  & 10.71415  & 8.669586  & 5.200832  & 6.820946  & 0.9630263 & 9.242116 \\
\end{tabular}


    
    \(y_i\), i. gözlemi, \(\hat{y}_i\), i. gözlem için tahmini temsil etmek
üzere hata terimi aşağıdaki şekilde ifade edilir. \[e_i=y_i-\hat{y}_i\]

\paragraph{Ölçeğe bağımlı
hatalar}\label{uxf6luxe7eux11fe-baux11fux131mlux131-hatalar}

Bu hata hesaplamalarında serilerin benzer ölçekte olmaları
gerekmektedir.Doğrudan \(e_t\)'ye bağlı hata ölçümü olduğundan farklı
ölçeklerdeki serilerin karşılaştırılmasında kullanılamaz. En yaygın
kullanılan ölçüm parametrelerinden bazıları mutlak hatalara veya karesel
hatalara dayanmaktadır.

Ortalama Mutlak Hata: \[MAE= mean(|e_i|)\]

Ortalama Kareler Hata: \[RMSE=\sqrt{mean(e_i)^2}\]

Tahmin yöntemlerini tek bir veri kümesinde karşılaştırırken, MAE,
anlaşılması ve hesaplanması kolay olduğu için daha çok tercih
edilmektedir.

\paragraph{Yüzdeye bağımlı
hatalar}\label{yuxfczdeye-baux11fux131mlux131-hatalar}

Yüzde hatası \(p_t = 100\frac{e_t}{y_t}\) olarak ifade edilir. Yüzde
hataları serilerin ölçeklerinden bağımsız olduklarından ölçek farklı
seriler arasındaki hata ölçümü olarak kullanılmaktadır. En sık
kullanılan mutlak yüzde hatası MAPE aşağıdaki şekilde ifade
edilmektedir.

Ortalama Mutlak Yüzde Hatası: \[MAPE= mean(|p_t|)\]

Accuracy() foksiyonlarının çıktısına göre Arima ile yapılan
tahminlemenin MAE,RMSE ve MAPE gibi hata çıktılarına göre daha yakın
sonuçlar verdiği gözlenmiştir.Buna göre Arima modeli ile tüm verilerin
kullanılarak sonraki 30 gün için tahminleme yapılmıştır.

    \begin{Verbatim}[commandchars=\\\{\}]
{\color{incolor}In [{\color{incolor}23}]:} PlatinumPricesArimaForecast\PY{o}{\PYZlt{}\PYZhy{}}arima\PY{p}{(}na.omit\PY{p}{(}PlatinumPricests\PY{p}{)}\PY{p}{,}
                                            order\PY{o}{=}\PY{k+kt}{c}\PY{p}{(}\PY{l+m}{1}\PY{p}{,}\PY{l+m}{0}\PY{p}{,}\PY{l+m}{3}\PY{p}{)}\PY{p}{)}
         tahmin \PY{o}{\PYZlt{}\PYZhy{}} forecast\PY{p}{(}PlatinumPricesArimaForecast\PY{p}{,}\PY{l+m}{30}\PY{p}{)}
         plot\PY{p}{(}tahmin\PY{p}{)}
\end{Verbatim}


    \begin{center}
    \adjustimage{max size={0.9\linewidth}{0.9\paperheight}}{output_42_0.png}
    \end{center}
    { \hspace*{\fill} \\}
    
    Uygun yöntem ile 30 günlük tahminleme yapıldıktan sonra başka bir analiz
olan çok değişkenli zaman serilerinin analizi incelenecektir.

\section{Çok Değişkenli Zaman Serileri
Analizi}\label{uxe7ok-deux11fiux15fkenli-zaman-serileri-analizi}

Veriler genellikle birden fazla değişken üzerinde toplanır. Örneğin,
ekonomide, günlük döviz kurları çok çeşitli para birimleri için
kullanılabilir, ya da hidrolojik çalışmalarda, hem yağış hem de nehir
akış ölçümleri ilgili bir alanda alınabilir. Zamanla ölçülen değişkenler
genellikle benzer özellikler sergilediğinden, değişkenleri
ilişkilendirmek için regresyon kullanılabilir. Bununla birlikte, zaman
serileri değişkenlerinin regresyon modelleri yanıltıcı olabilmektedir.
Bu durum sahte regresyon olarak adlandırılır.Zaman serileri değişkenleri
için, nedensel ilişkiyi ortaya çıkarmadan önce dikkatli olunmalıdır,
çünkü zaman serisinin içerdiği anomaliler nedeniyle belirgin bir ilişki
görülebilir. Örneğin artan nüfus ile artan birbiriyle ilgili olmayan iki
malın miktarı arasında ilişki gözlenebilir.

\subsection{Dağıtılmış-Gecikmeli
Model}\label{daux11fux131tux131lmux131ux15f-gecikmeli-model}

Dağıtılmış-gecikmeli bir model, X'in y üzerindeki bir seferlik etkisi
değil zaman içinde meydana getirdiği etkinin dinamik modelidir.
\[y_t=\alpha + \sum_{s=0}^{\infty}\beta_sx_{t-s}+u_t\]

\(u_t\), durağan bir hata terimidir. Bireysel katsayılar \(\beta_s\),
gecikme ağırlıkları olarak adlandırılır ve toplu olarak gecikme
dağılımını verir. X'in zamanla y'yi nasıl etkilediğini tanımlar.
Denklemdeki sonsuz sayıda \(\beta\) katsayılarını tahmin edilmesi
zordur. Pratik bir yöntem, gecikle dağılımı etkin bir şekilde sıfır
olduğunda uygun olan gecikmeyi sonlu uzunluğa (q) kesmektir.Başka bir
yaklaşım, denklemde gecikme dağılımının kademeli olarak sıfıra düşmesine
izin veren fonksiyonel bir formun kullanılmasıdır.

\subsection{Kovaryans}\label{kovaryans}

Kovaryans, iki değişken arasında herhangi bir ilişki olup olmadığını
gösterir. Pozitif bir kovaryans değişkenler arasında pozitif bir
doğrusal ilişki olduğunu ve negatif kovaryans ise negatif bir ilişkinin
varlığını gösterir.Kovaryans aşağıdaki formül ile hesaplanır.
\[ Cov_{xy} = \frac {\sum{(x_i-\bar{x})(y_i-\bar{y})}}{n-1} \]

\subsection{Korelasyon}\label{korelasyon}

Bir korelasyon, iki rastgele değişken arasındaki ilişkinin yönlü
ölçüsüdür. Korelasyon iki değişken arasında tamamen simetriktir.
Korelasyon analizi ile bağımsız değişken değiştiğinde,bağımlı değişkenin
nasıl değişeceği saptanır. Korelasyon katsayısı -1 ile +1 arasında
değerler alır. Korelasyon aşağıdaki formül ile ifade edilir. Buradaki
\(s_x,s_y\), x ve y'nin standart sapmasıdır.
\[ Cor_{xy} = \frac {Cov_{xy} }{s_x s_y} \]

Çok değişkenli zaman serisi modellerinden çapraz korelasyon fonksiyonu
ile açıklanacaktır.

\subsubsection{Çapraz Korelasyon}\label{uxe7apraz-korelasyon}

Çok değikenli zaman serilerinde ele alacağımız temel amaç, iki zaman
dizisi arasındaki ilişkinin açıklaması ve modellenmesidir.İki zaman
dizisi (\(y_t\) ve \(x_t\)) arasındaki ilişkide, \(y_t\) serisi, \(x_t\)
serisinin geçmiş gecikmeleriyle ilişkili olabilir. Çapraz korelasyon
fonksiyonu (CCF),\(y_t\)'nin yararlı belirleyicileri olabilecek \(x_t\)
değişkeninin gecikme sürelerinin belirlenmesinde yardımcı olacaktır.
R'de bulunan çapraz korelasyon fonksiyonu(CCF), \(x_{t + h}\) ve \(y_t\)
arasında h = 0, ± 1, ± 2, ± 3 ve benzeri için örnek korelasyonları
kümesi olarak tanımlanır. h için negatif bir değer, t'den önceki bir
zamanda \(x\) değişkeni ve t zamanında \(y\) değişkeni arasında bir
korelasyondur. Örneğin, h = −2'yi düşündüğümüzde CCF değeri, \(x_{t-2}\)
ve \(y_t\) arasındaki korelasyonu vermektedir.

Öncelikle olarak Platin fiyatlarının genel olarak birbirleriyle
aralarındaki ilişki incelenecektir. Korelasyon rastgele iki değişken
arasındaki ilişkinin gücünü ve yönünü belirtmek için kullanılır.
Platinum fiyatları çoklu değişkene sahip olduğu için bu ilişkiyi bir
matris ile tanımlayacağız. Korelasyon matrisi adı verilecek olan bu
matris, çoklu değişkenler arasındaki korelasyon katsayılarını gösterir.
Bu matris yardımı ile iki değişken arasında korelasyon kolaylıkla
görülebilir. Daha sonra elde edilen bu matris R'da corrplot fonksiyonu
ile görselleştirilebilir.{[}3{]}

    \begin{Verbatim}[commandchars=\\\{\}]
{\color{incolor}In [{\color{incolor}24}]:} cormat\PY{o}{\PYZlt{}\PYZhy{}}cor\PY{p}{(}na.omit\PY{p}{(}PlatinumPrices\PY{p}{[}\PY{p}{,}\PY{k+kt}{c}\PY{p}{(}\PY{l+m}{2}\PY{o}{:}\PY{l+m}{7}\PY{p}{)}\PY{p}{]}\PY{p}{)}\PY{p}{)}
         cormat
         corrplot\PY{p}{(}cormat\PY{p}{,} method \PY{o}{=} \PY{l+s}{\PYZdq{}}\PY{l+s}{color\PYZdq{}}\PY{p}{)}
\end{Verbatim}


    \begin{tabular}{r|llllll}
  & USD AM & EUR AM & GBP AM & USD PM & EUR PM & GBP PM\\
\hline
	USD AM & 1.0000000 & 0.9404636 & 0.9800171 & 0.9997624 & 0.9404015 & 0.9796666\\
	EUR AM & 0.9404636 & 1.0000000 & 0.9559812 & 0.9401127 & 0.9997532 & 0.9555518\\
	GBP AM & 0.9800171 & 0.9559812 & 1.0000000 & 0.9799510 & 0.9560487 & 0.9997825\\
	USD PM & 0.9997624 & 0.9401127 & 0.9799510 & 1.0000000 & 0.9404793 & 0.9800054\\
	EUR PM & 0.9404015 & 0.9997532 & 0.9560487 & 0.9404793 & 1.0000000 & 0.9560112\\
	GBP PM & 0.9796666 & 0.9555518 & 0.9997825 & 0.9800054 & 0.9560112 & 1.0000000\\
\end{tabular}


    
    \begin{center}
    \adjustimage{max size={0.9\linewidth}{0.9\paperheight}}{output_44_1.png}
    \end{center}
    { \hspace*{\fill} \\}
    
    Tabloda görüldüğü üzere aynı para birimi cinsinden açılış fiyatları ile
kapanış fiyatları ilişkilidir.Ayrıca tabloda görüldüğü üzere herhangi
bir para birimi cinsinden açılış fiyatları başka bir açılış fiyatıyla,
kapanış fiyatına göre daha çok ilişkilidir. Grafikte ise tablonun
görselleştirilmiş hali görülmektedir. Mavi tonları koyulaştıkça
değişkenler arasındaki ilişki pozitif yönde artış göstermektedir.

Genel olarak para birimleri arasındaki ilişki incelendikten sonra dolar
cinsinden açılış ve kapanış fiyatlarının arasındaki ilişki
saptanacaktır. Aşağıda öncelikle aralarındaki korelasyonu
inceleyeceğimiz dolar cinsinden cuma günü kapanış fiyatı ile pazartesi
günü açılış fiyatlarının verimizden ayrıştırılması gerekmektedir. Bunun
için aşağıdaki R kodları kullanılmaktadır.

    \begin{Verbatim}[commandchars=\\\{\}]
{\color{incolor}In [{\color{incolor}25}]:} \PY{k+kr}{for} \PY{p}{(}i \PY{k+kr}{in} \PY{l+m}{1}\PY{o}{:}\PY{l+m}{7}\PY{p}{)}\PY{p}{\PYZob{}}
         PlatinumPrices\PY{p}{[}\PY{p}{,}i\PY{p}{]}\PY{o}{\PYZlt{}\PYZhy{}}\PY{k+kp}{rev}\PY{p}{(}PlatinumPrices\PY{p}{[}\PY{p}{,}i\PY{p}{]}\PY{p}{)}
             \PY{p}{\PYZcb{}}
         PlatinumPricesDolarAM\PY{o}{\PYZlt{}\PYZhy{}}PlatinumPrices\PY{p}{[}\PY{p}{,}\PY{l+m}{2}\PY{p}{]}
         PlatinumPricesDolarPM\PY{o}{\PYZlt{}\PYZhy{}}PlatinumPrices\PY{p}{[}\PY{p}{,}\PY{l+m}{5}\PY{p}{]}
         Monday\PY{o}{\PYZlt{}\PYZhy{}}\PY{k+kt}{c}\PY{p}{(}\PY{k+kc}{TRUE}\PY{p}{,}\PY{k+kc}{FALSE}\PY{p}{,}\PY{k+kc}{FALSE}\PY{p}{,}\PY{k+kc}{FALSE}\PY{p}{,}\PY{k+kc}{FALSE}\PY{p}{)}
         Friday\PY{o}{\PYZlt{}\PYZhy{}}\PY{k+kt}{c}\PY{p}{(}\PY{k+kc}{FALSE}\PY{p}{,}\PY{k+kc}{FALSE}\PY{p}{,}\PY{k+kc}{FALSE}\PY{p}{,}\PY{k+kc}{FALSE}\PY{p}{,}\PY{k+kc}{TRUE}\PY{p}{)}
         n\PY{o}{\PYZlt{}\PYZhy{}}\PY{k+kp}{length}\PY{p}{(}PlatinumPricesDolarPM\PY{p}{)}
         P\PY{o}{\PYZlt{}\PYZhy{}}\PY{k+kp}{rep}\PY{p}{(}Monday\PY{p}{,}n\PY{o}{/}\PY{l+m}{5}\PY{p}{)}
         C\PY{o}{\PYZlt{}\PYZhy{}}\PY{k+kp}{rep}\PY{p}{(}Friday\PY{p}{,}n\PY{o}{/}\PY{l+m}{5}\PY{p}{)}
         PAM\PY{o}{\PYZlt{}\PYZhy{}}\PY{k+kp}{subset}\PY{p}{(}PlatinumPricesDolarAM\PY{p}{,}P\PY{p}{[}\PY{k+kc}{TRUE}\PY{p}{]}\PY{p}{)}
         PPM\PY{o}{\PYZlt{}\PYZhy{}}\PY{k+kp}{subset}\PY{p}{(}PlatinumPricesDolarPM\PY{p}{,}C\PY{p}{[}\PY{k+kc}{TRUE}\PY{p}{]}\PY{p}{)}
\end{Verbatim}


    Aralarındaki korelasyona bakacağımız cuma kapanış fiyatı ile pazartesi
açılış fiyatları çarpraz korelasyon R'daki ccf() fonksiyonu yardımı ile
bulunabilir.

    \begin{Verbatim}[commandchars=\\\{\}]
{\color{incolor}In [{\color{incolor}26}]:} ccfvalues \PY{o}{\PYZlt{}\PYZhy{}} ccf \PY{p}{(}na.omit\PY{p}{(}PAM\PY{p}{)}\PY{p}{,} na.omit\PY{p}{(}PPM\PY{p}{)}\PY{p}{,}lag\PY{o}{=}\PY{l+m}{20}\PY{p}{)}
         ccfvalues
\end{Verbatim}


    
    \begin{verbatim}

Autocorrelations of series 'X', by lag

  -20   -19   -18   -17   -16   -15   -14   -13   -12   -11   -10    -9    -8 
0.851 0.858 0.864 0.871 0.878 0.884 0.890 0.897 0.903 0.909 0.915 0.921 0.927 
   -7    -6    -5    -4    -3    -2    -1     0     1     2     3     4     5 
0.933 0.940 0.946 0.953 0.959 0.966 0.972 0.978 0.980 0.981 0.982 0.982 0.980 
    6     7     8     9    10    11    12    13    14    15    16    17    18 
0.978 0.976 0.974 0.970 0.965 0.960 0.954 0.947 0.942 0.936 0.930 0.923 0.917 
   19    20 
0.910 0.904 
    \end{verbatim}

    
    \begin{center}
    \adjustimage{max size={0.9\linewidth}{0.9\paperheight}}{output_48_1.png}
    \end{center}
    { \hspace*{\fill} \\}
    
    Yukarıdaki grafikten anlaşılacağı üzere en yüksek çapraz korelasyon
katsayıları -5 ile 0 arasında meydana gelmektedir. Gecikmeleri tam
olarak grafikten okumak zordur, bu yüzden ccf'nin listelenmesi daha
doğru sonuçlar verecektir. Yukarıda yapılan listelemeye göre cuma
kapanış fiyatı üç,dört gün sonraki pazartesi açılış fiyatından
etkilenmektedir.
\pagebreak

    \section{Değerlendirme}\label{deux11ferlendirme}

Bir zaman serisi, değişkenlerin zaman içinde belirli sabit bir aralıkta
ölçüldüğünde, elde edilen veriler ile oluşturmaktadır. Zaman
serilerindeki analizin amacı; geçmiş verilerdeki anomalileri saptamak,
birbirleriyle ilişkilerini gözlemlemek ve gelecek için tahminleme
yapabilmektir. Anomalilerin çıkarılmasının nedeni verideki anormalilerin
çok çeşitli uygulama alanlarında önemli ve eyleme geçirilebilir
bilgilere dönüşebilir olmasıdır.

Analize başlamadan önce ilk olarak veri seti uygun şekilde yüklenmiş ve
zaman serisine dönüştürülmüştür. Daha sonra verinin özeti oluşturulmuş
ve grafiği çizilerek verinin anlamlandırılması sağlanmıştır.

Zaman serisi analizindeki ilk aşamada verinin içindeki anomaliler
saptanmış ve çıkartılmıştır. Bunun için bir dizi yöntemler
uygulanmıştır. Öncelikle bu aşamada mevsimsellik ve trend gibi
bileşenlerin çıkarılmıştır. Mevsimsellik ve trend bileşenlerinin veriden
uzaklaştırmak için ayrıştırma işlemleri yapılmıştır. Bu ayrıştırma
yöntemleri; toplamsal ve çarpımsal yöntem ile STL ile ayrıştırma
yöntemidir. Bu üç ayrıştırma yöntemi veriye uygulandıktan sonra kalanın
gürültü verisine uygun olması beklenir. Gürültü, sıfır ortalamalı
gaussian (normal) dağılıma sahip rastgele değişkendir. Fakat bu
yöntemlerden kalan verinin gürültüden farklı olduğu istatiksel
yöntemlerle gösterilmiştir. Bu da veri setinin içinde hala bazı
sinyallerinin olduğunu göstermektedir.Veri seti içindeki tüm sinyallerin
çıkarılması için Box-Jenkins yöntemleri uygulanmıştır.

Ayrıştırma yöntemleriyle verinin içindeki üssel dağılıma sahip bazı
sinyaller çıkarılmıştır. Fakat arima modeli uygulanmadan önce verinin
durağanlığı istatiksel yöntemlerle test edilmiştir. Bu istatiksel
yöntemler sonucunda verinin durağan olduğu sonucuna varılmıştır. Daha
sonra uygun arima modelinin AIC,BIC değerlerinin küçük olan ARIMA(1,0,3)
modeli olduğu saptanmıştır. ARIMA(1,0,3) modelinden kalan veriye
normallik testleri uygulanmıştır ve sinyaller çıkarıldıktan sonra
verinin gürültü olduğu istatiksel yöntemlerle bulunmuştur.

Analizin ikinci aşamasında tahminleme işlemi yapılmıştır. Tahminleme
için arima ve Holt-Winters yöntemleri kullanılmıştır. Yapılan bu iki
tahminlemenin istatiksel olarak ortalama kareler hatası ve ortalama
mutlak hatası sonuçlarına göre iyi sonuçlar veren modeli olan ARIMA
modeli seçilmiştir. Bu arima modeli kullanılarak önümüzdeki 30 gün için
tahminleme yapılmıştır.

Analizdeki üçüncü aşamasında çok değişkenli zaman serileri için
korelasyon ve kovaryans incelenmiştir. Veri setimizdeki platin
fiyatlarının tüm para birimleri acısından açılış ve kapanış fiyatları
arasındaki korelasyon değerleri bulunmuştur. Daha sonra cuma kapanış
fiyatları ile pazartesi açılış fiyatları arasındaki korelasyon
incelenmiş ve cuma kapanış fiyatı üç,dört gün sonraki pazartesi açılış
fiyatından etkilendiği saptanmıştır.
\pagebreak
    \section{Kaynakça}\label{kaynakuxe7a}

{[}1{]} Paul S.P. Cowpertwait · Andrew V. Metcalfe (2005). Introductory
Time Series with R

{[}2{]} Peter J. Brockwell, Richard A. Davis (2002). Introduction to
Time Series and Forecasting

{[}3{]} Dr. Marcel Dettling (2014). Applied Time Series Analysis

{[}4{]} Peter J. Brockwell, Richard A. Davis (2009). Time Series: Theory
and Methods

{[}5{]} Christian Kleiber, Achim Zeileis (2008) Applied Econometrics
with R

{[}6{]} A. Ian McLeod, Hao Yu, Esam Mahdi (2011) Time Series Analysis
with R

{[}7{]} Robert H. Shumway, David S. Stoffer (2011) Time Series Analysis
and Its Applications


    % Add a bibliography block to the postdoc
    
    
    
    \end{document}
